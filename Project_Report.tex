%%%%%%%%%%%%  Generated using docx2latex.com  %%%%%%%%%%%%%%

%%%%%%%%%%%%  v2.0.0-beta  %%%%%%%%%%%%%%

\documentclass[12pt]{report}
\usepackage{amsmath}
\usepackage{latexsym}
\usepackage{amsfonts}
\usepackage[normalem]{ulem}
\usepackage{soul}
\usepackage{array}
\usepackage{amssymb}
\usepackage{extarrows}
\usepackage{graphicx}
\usepackage[backend=biber,
style=numeric,
sorting=none,
isbn=false,
doi=false,
url=false,
]{biblatex}\addbibresource{bibliography.bib}

\usepackage{subfig}
\usepackage{wrapfig}
\usepackage{wasysym}
\usepackage{enumitem}
\usepackage{adjustbox}
\usepackage{ragged2e}
\usepackage[svgnames,table]{xcolor}
\usepackage{tikz}
\usepackage{longtable}
\usepackage{changepage}
\usepackage{setspace}
\usepackage{hhline}
\usepackage{multicol}
\usepackage{tabto}
\usepackage{float}
\usepackage{multirow}
\usepackage{makecell}
\usepackage{fancyhdr}
\usepackage[toc,page]{appendix}
\usepackage[hidelinks]{hyperref}
\usetikzlibrary{shapes.symbols,shapes.geometric,shadows,arrows.meta}
\tikzset{>={Latex[width=1.5mm,length=2mm]}}
\usepackage{flowchart}\usepackage[paperheight=11.69in,paperwidth=8.27in,left=0.89in,right=0.04in,top=0.18in,bottom=0.43in,headheight=1in]{geometry}
\usepackage[utf8]{inputenc}
\usepackage[T1]{fontenc}
\TabPositions{0.5in,1.0in,1.5in,2.0in,2.5in,3.0in,3.5in,4.0in,4.5in,5.0in,5.5in,6.0in,6.5in,7.0in,}

\urlstyle{same}

\renewcommand{\_}{\kern-1.5pt\textunderscore\kern-1.5pt}

 %%%%%%%%%%%%  Set Depths for Sections  %%%%%%%%%%%%%%

% 1) Section
% 1.1) SubSection
% 1.1.1) SubSubSection
% 1.1.1.1) Paragraph
% 1.1.1.1.1) Subparagraph


\setcounter{tocdepth}{5}
\setcounter{secnumdepth}{5}


 %%%%%%%%%%%%  Set Depths for Nested Lists created by \begin{enumerate}  %%%%%%%%%%%%%%


\setlistdepth{9}
\renewlist{enumerate}{enumerate}{9}
		\setlist[enumerate,1]{label=\arabic*)}
		\setlist[enumerate,2]{label=\alph*)}
		\setlist[enumerate,3]{label=(\roman*)}
		\setlist[enumerate,4]{label=(\arabic*)}
		\setlist[enumerate,5]{label=(\Alph*)}
		\setlist[enumerate,6]{label=(\Roman*)}
		\setlist[enumerate,7]{label=\arabic*}
		\setlist[enumerate,8]{label=\alph*}
		\setlist[enumerate,9]{label=\roman*}

\renewlist{itemize}{itemize}{9}
		\setlist[itemize]{label=$\cdot$}
		\setlist[itemize,1]{label=\textbullet}
		\setlist[itemize,2]{label=$\circ$}
		\setlist[itemize,3]{label=$\ast$}
		\setlist[itemize,4]{label=$\dagger$}
		\setlist[itemize,5]{label=$\triangleright$}
		\setlist[itemize,6]{label=$\bigstar$}
		\setlist[itemize,7]{label=$\blacklozenge$}
		\setlist[itemize,8]{label=$\prime$}



 %%%%%%%%%%%%  Header here  %%%%%%%%%%%%%%


\pagestyle{fancy}
\fancyhf{}
\cfoot{ 
\vspace{\baselineskip}
}
\renewcommand{\headrulewidth}{0pt}
\setlength{\topsep}{0pt}\setlength{\parindent}{0pt}
\renewcommand{\arraystretch}{1.3}


%%%%%%%%%%%%%%%%%%%% Document code starts here %%%%%%%%%%%%%%%%%%%%



\begin{document}

\vspace{\baselineskip}
\setstretch{2.1}
\begin{adjustwidth}{0.2in}{0.84in}
\begin{Center}
{\fontsize{22pt}{26.4pt}\selectfont \textbf{\textcolor[HTML]{0D0D0D}{DECLARATION}}\par}
\end{Center}\par

\end{adjustwidth}


\vspace{\baselineskip}

\vspace{\baselineskip}
\setstretch{1}
\begin{adjustwidth}{0.2in}{0.84in}
\begin{justify}
\textcolor[HTML]{0D0D0D}{We \textbf{Yash Borle, Shubham Khandelwal, Aayush Chourasia, Aman Kumar, Department of Information Technology }declare that the dissertation \textbf{$``$ERP- Stationary Management$"$  }is our own work conducted under the supervision of Mr. \textbf{Upendra Singh Assistant Professor, Information Technology Department, S.G.S.I.T.S. Indore (M.P.).}}
\end{justify}\par

\end{adjustwidth}

\begin{adjustwidth}{0.2in}{0.84in}
\textcolor[HTML]{0D0D0D}{We further declare that to the best of our knowledge this dissertation work does not contain any part of any work which has been submitted for the award of any degree or any other work either in this University or in any other University/ websites without proper citation.}\par

\end{adjustwidth}


\vspace{\baselineskip}

\vspace{\baselineskip}

\vspace{\baselineskip}

\vspace{\baselineskip}

\vspace{\baselineskip}
\begin{adjustwidth}{0.2in}{0.84in}
\textbf{\textcolor[HTML]{0D0D0D}{Signature of the candidates:-}}\par

\end{adjustwidth}


\vspace{\baselineskip} \tabto{2.69in} 
\vspace{\baselineskip}\begin{adjustwidth}{0.2in}{0.84in}
\textbf{\textcolor[HTML]{0D0D0D}{Name and Enrollment no. of the candidates:-\ \ \ \  Yash Borle (0801IT183D18)}}\par

\end{adjustwidth}

\begin{adjustwidth}{3.0in}{0.84in}
\textcolor[HTML]{0D0D0D}{ \tabto{2.66in} \  Shubham Khandelwal (0801IT171081)}\par

\end{adjustwidth}

\begin{adjustwidth}{0.2in}{0.84in}
 \tabto{2.66in}  \tabto{2.93in} \tab \textcolor[HTML]{0D0D0D}{\tab \  Ayush Chourasia (0801IT183D01)}\par

\end{adjustwidth}

\begin{adjustwidth}{0.2in}{0.84in}
 \tabto{2.66in}  \tabto{2.93in} \tab \textcolor[HTML]{0D0D0D}{\tab \  Aman Kumar (0801IT183D04)}\par

\end{adjustwidth}


\vspace{\baselineskip}\begin{adjustwidth}{0.2in}{0.84in}
\textbf{\textcolor[HTML]{0D0D0D}{Date:-  \tabto{2.61in} \tab \  \tab \  15/04/2020}}\par

\end{adjustwidth}


\vspace{\baselineskip}
\setstretch{2.1}

\vspace{\baselineskip}
\begin{adjustwidth}{0.2in}{0.84in}
\begin{Center}
{\fontsize{22pt}{26.4pt}\selectfont \textbf{\textcolor[HTML]{0D0D0D}{ACKNOWLEDGEMENT}}\par}
\end{Center}\par

\end{adjustwidth}


\vspace{\baselineskip}

\vspace{\baselineskip}
\setstretch{1}
\begin{adjustwidth}{0.2in}{0.84in}
\textcolor[HTML]{0D0D0D}{With great pleasure and sense of obligation we express my heartfelt gratitude to my esteemed guide \textbf{Mr. Upendra Singh, Assistant Professor, Department of Information Technology, S.G.S.I.T.S. Indore, whose constant encouragement enabled me to work enthusiastically. My project guide, in spite of his heavy work commitments and busy schedule, have been there for his invaluable guidance and ever ready support. His persistent encouragement, perpetual motivation, everlasting patience and excellent expertise in discussions, during progress of the work, have benefited to an extent which is beyond expression. His contributions are beyond the purview of the acknowledgement.}}\par

\end{adjustwidth}

\begin{adjustwidth}{0.2in}{0.84in}
\textcolor[HTML]{0D0D0D}{We would like to give warm expression of thanks to \textbf{Dr. Rakesh Saxena, Director, S.G.S.I.T.S. Indore, for providing all the facilities and academic environment during the course of study.}}\par

\end{adjustwidth}

\begin{adjustwidth}{0.2in}{0.84in}
\textcolor[HTML]{0D0D0D}{We sincerely wish to express, our gratefulness to all the members of staff of Computer Engineering Department who have extended their cooperation at all times and have contributed in their own way in developing the project.}\par

\end{adjustwidth}

\begin{adjustwidth}{0.2in}{0.84in}
\textcolor[HTML]{0D0D0D}{We are thankful to our parents for being a constant source of encouragement in all our endeavors. The support of our friends is worth appreciation and thankfulness.}\par

\end{adjustwidth}


\vspace{\baselineskip}

\vspace{\baselineskip}

\vspace{\baselineskip}

\vspace{\baselineskip}

\vspace{\baselineskip}

\vspace{\baselineskip}

\vspace{\baselineskip}
\begin{adjustwidth}{0.2in}{0.84in}
\begin{Center}
{\fontsize{22pt}{26.4pt}\selectfont \textbf{\textcolor[HTML]{0D0D0D}{ABSTRACT}}\par}
\end{Center}\par

\end{adjustwidth}


\vspace{\baselineskip}

\vspace{\baselineskip}
\begin{adjustwidth}{0.2in}{0.84in}
\textcolor[HTML]{0D0D0D}{Every college has a stationary department which supplies all the required stationary items to the departments. It, itself has the high importance in college. Teachers also need these stationary items, hence have to update all the data about the requirements and the quantity they received.}\par

\end{adjustwidth}

\begin{adjustwidth}{0.2in}{0.84in}
\textcolor[HTML]{0D0D0D}{Every teacher in the college has a number works to do. But in their busy schedule if they are asked about their last four requirements, it will definitely consume time and it is possible that the requirements doesn’t tally.}\par

\end{adjustwidth}

\begin{adjustwidth}{0.2in}{0.84in}
\textcolor[HTML]{0D0D0D}{Therefore to decrease the work of the teachers and store manager software application or web page for the stationary department of the college. To bring in an easy user interface and interactive application which provides feasible options to everyone associated with the application directly and indirectly. }\par

\end{adjustwidth}


\vspace{\baselineskip}

\vspace{\baselineskip}


%%%%%%%%%%%%%%%%%%%% Table No: 1 starts here %%%%%%%%%%%%%%%%%%%%


{
\setlength\extrarowheight{3pt}
\begin{longtable}{p{5.88in}p{0.08in}}

\endfirsthead
\multicolumn{2}{c}{\textit{continued from previous page}}\hline
\endhead
\multicolumn{2}{r}{\textit{continued on next page}} \\
\endfoot
\endlastfoot%row no:1
\multicolumn{1}{p{5.88in}}{{\fontsize{17pt}{20.4pt}\selectfont \textbf{\textcolor[HTML]{0D0D0D}{Table of Contents}}}} & 
\multicolumn{1}{p{0.08in}}{\multirow{1}{*}{\begin{tabular}{p{0.08in}}\end{tabular}}} \\ 

\hhline{~~}
%row no:2
\multicolumn{1}{p{5.88in}}{\textbf{\textcolor[HTML]{0D0D0D}{Recommendation}}} & 
\multicolumn{1}{p{0.08in}}{} \\
\hhline{~~}
%row no:3
\multicolumn{1}{p{5.88in}}{\textbf{\textcolor[HTML]{0D0D0D}{Certificate}}} & 
\multicolumn{1}{p{0.08in}}{} \\
\hhline{~~}
%row no:4
\multicolumn{1}{p{5.88in}}{\textbf{\textcolor[HTML]{0D0D0D}{Declaration}}} & 
\multicolumn{1}{p{0.08in}}{} \\
\hhline{~~}
%row no:5
\multicolumn{1}{p{5.88in}}{\textbf{\textcolor[HTML]{0D0D0D}{Acknowledgement}}} & 
\multicolumn{1}{p{0.08in}}{} \\
\hhline{~~}
%row no:6
\multicolumn{1}{p{5.88in}}{\textbf{\textcolor[HTML]{0D0D0D}{Abstract}}} & 
\multicolumn{1}{p{0.08in}}{} \\
\hhline{~~}
%row no:7
\multicolumn{1}{p{5.88in}}{\textbf{\textcolor[HTML]{0D0D0D}{List of Figures}}} & 
\multicolumn{1}{p{0.08in}}{} \\
\hhline{~~}
%row no:8
\multicolumn{1}{p{5.88in}}{\begin{enumerate}
	\item \textbf{\textcolor[HTML]{0D0D0D}{INTRODUCTION}}
\end{enumerate}} & 
\multicolumn{1}{p{0.08in}}{\textbf{\textcolor[HTML]{0D0D0D}{1}}} \\
\hhline{~~}
%row no:9
\multicolumn{1}{p{5.88in}}{\textcolor[HTML]{0D0D0D}{\ \ \ \ \ \ \ 1.1.\   Preamble}} & 
\multicolumn{1}{p{0.08in}}{\textcolor[HTML]{0D0D0D}{1}} \\
\hhline{~~}
%row no:10
\multicolumn{1}{p{5.88in}}{\textcolor[HTML]{0D0D0D}{\ \ \ \ \ \ \ 1.2\ \   Need of the Project}} & 
\multicolumn{1}{p{0.08in}}{\textcolor[HTML]{0D0D0D}{1}} \\
\hhline{~~}
%row no:11
\multicolumn{1}{p{5.88in}}{\textcolor[HTML]{0D0D0D}{1.3 \tabto{0.65in} Problem Statement }} & 
\multicolumn{1}{p{0.08in}}{\textcolor[HTML]{0D0D0D}{1}} \\
\hhline{~~}
%row no:12
\multicolumn{1}{p{5.88in}}{\textcolor[HTML]{0D0D0D}{1.4 \tabto{0.65in} Project Objectives \tabto{2.13in} }} & 
\multicolumn{1}{p{0.08in}}{\textcolor[HTML]{0D0D0D}{1}} \\
\hhline{~~}
%row no:13
\multicolumn{1}{p{5.88in}}{\textcolor[HTML]{0D0D0D}{1.5 \tabto{0.65in} Solution Approach \tabto{2.13in} }} & 
\multicolumn{1}{p{0.08in}}{\textcolor[HTML]{0D0D0D}{1}} \\
\hhline{~~}
%row no:14
\multicolumn{1}{p{5.88in}}{\textbf{\textcolor[HTML]{0D0D0D}{2 \tabto{0.28in} BACKGROUND}}} & 
\multicolumn{1}{p{0.08in}}{\textbf{\textcolor[HTML]{0D0D0D}{2}}} \\
\hhline{~~}
%row no:15
\multicolumn{1}{p{5.88in}}{\textcolor[HTML]{0D0D0D}{2.1 \tabto{0.65in} Programming Language \tabto{2.51in} }} & 
\multicolumn{1}{p{0.08in}}{\textcolor[HTML]{0D0D0D}{2}} \\
\hhline{~~}
%row no:16
\multicolumn{1}{p{5.88in}}{\textcolor[HTML]{0D0D0D}{2.2 \tabto{0.65in} IDE \tabto{1.1in} }} & 
\multicolumn{1}{p{0.08in}}{\textcolor[HTML]{0D0D0D}{2}} \\
\hhline{~~}
%row no:17
\multicolumn{1}{p{5.88in}}{\textcolor[HTML]{0D0D0D}{2.3 \tabto{0.65in} Errors }} & 
\multicolumn{1}{p{0.08in}}{\textcolor[HTML]{0D0D0D}{2}} \\
\hhline{~~}
%row no:18
\multicolumn{1}{p{5.88in}}{\textbf{\textcolor[HTML]{0D0D0D}{3 \tabto{0.28in} LITERATURE REVIEW}}} & 
\multicolumn{1}{p{0.08in}}{\textbf{\textcolor[HTML]{0D0D0D}{14}}} \\
\hhline{~~}
%row no:19
\multicolumn{1}{p{5.88in}}{\textcolor[HTML]{0D0D0D}{3.1 \tabto{0.65in} Summary of Reviewed Literature }} & 
\multicolumn{1}{p{0.08in}}{\textcolor[HTML]{0D0D0D}{14}} \\
\hhline{~~}
%row no:20
\multicolumn{1}{p{5.88in}}{\textbf{\textcolor[HTML]{0D0D0D}{4 \tabto{0.28in} ANALYSIS}} \par 

 %%%%%%%%%%%%  This Produces Table Of Contents %%%%%%%%%%%%%%

\tableofcontents
\addcontentsline{toc}{chapter}{Contents}
 \par  \par  \par  \par  \par \textcolor[HTML]{0D0D0D}{\ \ \ \ \ \ \ \ \  4.6\  }} & 
\multicolumn{1}{p{0.08in}}{\textbf{\textcolor[HTML]{0D0D0D}{16}}\textcolor[HTML]{0D0D0D}{16} \par \textcolor[HTML]{0D0D0D}{17} \par \textcolor[HTML]{0D0D0D}{1818} \par \textcolor[HTML]{0D0D0D}{19} \par \textcolor[HTML]{0D0D0D}{19} \par } \\
\hhline{~~}

\end{longtable}}

%%%%%%%%%%%%%%%%%%%% Table No: 1 ends here %%%%%%%%%%%%%%%%%%%%


\vspace{\baselineskip}

\vspace{\baselineskip}
\vspace{\baselineskip}

\vspace{\baselineskip}

\vspace{\baselineskip}
\begin{adjustwidth}{3.7in}{0.84in}
{\fontsize{24pt}{28.8pt}\selectfont \textbf{\textcolor[HTML]{0D0D0D}{INTRODUCTION}}\par}\end{adjustwidth}


\vspace{\baselineskip}

\vspace{\baselineskip}
\vspace{\baselineskip}

\vspace{\baselineskip}
\begin{adjustwidth}{0.2in}{0.84in}
\textcolor[HTML]{0D0D0D}{In this chapter an overview of the project is given. Need of the project and problem statement are given and the solution approach has been explained. At the end, organization of the report has been given.}\par

\end{adjustwidth}


\vspace{\baselineskip}
\begin{enumerate}
	\item {\fontsize{14pt}{16.8pt}\selectfont \textbf{\textcolor[HTML]{0D0D0D}{Preamble}}\par}\par


\vspace{\baselineskip}
\textcolor[HTML]{0D0D0D}{The stationary department of every college during the management faces the following issues:}\par

\begin{enumerate}
	\item \textcolor[HTML]{0D0D0D}{Track of the requisites.}\par

	\item \textcolor[HTML]{0D0D0D}{Proper records.}\par

	\item \textcolor[HTML]{0D0D0D}{Authentication of old orders if required.}\par

	\item \textcolor[HTML]{0D0D0D}{Maintenance }\par

	\item \textcolor[HTML]{0D0D0D}{Time consumption}\par

\textcolor[HTML]{0D0D0D}{ }\par


\vspace{\baselineskip}

\end{enumerate}
	\item {\fontsize{14pt}{16.8pt}\selectfont \textbf{\textcolor[HTML]{0D0D0D}{Need of the Project}}\par}\par


\vspace{\baselineskip}
\begin{justify}
\textcolor[HTML]{0D0D0D}{Due to high load and work pressure it sometimes become tough to keep the records or to give the time by the teachers to authenticate the records. It sometimes take a no of days to get tally the records if done manually. Hence to reduce them this project is made.}
\end{justify}\par


\vspace{\baselineskip}
	\item {\fontsize{14pt}{16.8pt}\selectfont \textbf{\textcolor[HTML]{0D0D0D}{Problem Statement}}\par}\par


\vspace{\baselineskip}
\textcolor[HTML]{0D0D0D}{Most of the college maintain a register for recording the transactions related to the stationary products purchased for the college. Very few colleges use automated system software to carry out stationary products related activities.}\par


\vspace{\baselineskip}
	\item {\fontsize{14pt}{16.8pt}\selectfont \textbf{\textcolor[HTML]{0D0D0D}{Project Objectives}}\par}\par


\vspace{\baselineskip}
\textcolor[HTML]{0D0D0D}{Objective of project is to save time of faculty as well as the store manager. So in that time faculty can do their alot of work. To bring in an easy user interface and interactive application which provides feasible options to everyone associated with the application directly and indirectly.}\par


\vspace{\baselineskip}
	\item {\fontsize{14pt}{16.8pt}\selectfont \textbf{\textcolor[HTML]{0D0D0D}{Solution Approach}}\par}
\end{enumerate}\par


\vspace{\baselineskip}
\begin{adjustwidth}{0.2in}{0.84in}
\textcolor[HTML]{0D0D0D}{To provide a software application to carry out the tasks related to stationary department of the college. This application helps to record entries and transaction generate bills for the accounts section, make requisition to suppliers, etc. this application provides an easy user interface which helps the users to learn the usage of the application easily and quickly}{\fontsize{10pt}{12.0pt}\selectfont \textcolor[HTML]{0D0D0D}{.

 %%%%%%%%%%%%  Starting New Page here %%%%%%%%%%%%%%

\newpage
}\par}\par

\end{adjustwidth}


\vspace{\baselineskip}

\vspace{\baselineskip}
\begin{adjustwidth}{3.7in}{0.84in}
\begin{justify}
{\fontsize{24pt}{28.8pt}\selectfont \textbf{\textcolor[HTML]{0D0D0D}{BACKGROUND}}\par}
\end{justify}\par

\end{adjustwidth}


\vspace{\baselineskip}

\vspace{\baselineskip}
\vspace{\baselineskip}

\vspace{\baselineskip}

\vspace{\baselineskip}
\begin{itemize}
	\item {\fontsize{14pt}{16.8pt}\selectfont \textbf{\textcolor[HTML]{0D0D0D}{Programming Language}}\par}\par


\vspace{\baselineskip}
\textcolor[HTML]{0D0D0D}{To properly define this term — a markup language is a language that annotates text so that the computer can manipulate that text. Most markup languages are human-readable because the annotations are written in a way to distinguish them from the text itself. For example, with HTML, XML, and XHTML.}{\fontsize{11pt}{13.2pt}\selectfont \textcolor[HTML]{0D0D0D}{ \par}Here in this work, we are basically concentrating on HTML dialect.}\par


\vspace{\baselineskip}
	\item {\fontsize{14pt}{16.8pt}\selectfont \textbf{\textcolor[HTML]{0D0D0D}{IDE}}\par}\par


\vspace{\baselineskip}
\textcolor[HTML]{0D0D0D}{An IDE generally contains a code editor, a compiler or interpreter, and a computer program, accessed through one graphical computer program (GUI). The user writes and edits ASCII text file within the code editor. The compiler interprets the ASCII text file into a decipherable language that’s feasible for a pc. and therefore the computer program tests the computer code to resolve any problems or bugs.}\par

\textcolor[HTML]{0D0D0D}{There are many IDE Available like NetBeans, Eclipse, CodeBlocks, Visual Studio etc.}\par

\textcolor[HTML]{0D0D0D}{In this work, we are using Netbeans and Sublime Text for implementation.}\par


\vspace{\baselineskip}
	\item {\fontsize{14pt}{16.8pt}\selectfont \textbf{\textcolor[HTML]{0D0D0D}{Errors}}\par}\par


\vspace{\baselineskip}
\textcolor[HTML]{0D0D0D}{In computer programming — which includes Markup Languages — there are generally two types of errors with code:}\par


\vspace{\baselineskip}
\textbf{\textcolor[HTML]{0D0D0D}{Syntax errors — these errors usually involve a mistake that causes the computer to be unable to execute or compile the program properly (e.g., a missing brace or parenthesis).}}\par


\vspace{\baselineskip}
\textbf{\textcolor[HTML]{0D0D0D}{Logic errors — these errors arise when the code is syntactically correct but doesn’t do exactly what it was meant to do.}}\par


\vspace{\baselineskip}
\textcolor[HTML]{0D0D0D}{With most programming languages, the first kind of error is impossible to overlook. The code will refuse to compile or run until the error is fixed, and the compiler will usually alert the programmer to the error and the line number on which it can be found.}\par


\vspace{\baselineskip}
\textcolor[HTML]{0D0D0D}{This makes finding and fixing syntax errors much easier than logic errors, which result in those general head-scratching moments of, $``$Why isn’t it doing what I want?$"$  Validators, as you might expect, can only find syntax errors.}\par


\vspace{\baselineskip}
\textcolor[HTML]{0D0D0D}{Although HTML is a declarative Markup language rather than a procedural programming language, syntax errors may still occur. However, syntax errors in a web page do not commonly cause the web browser to refuse to display the page. This inherent forgiveness in web browsers is one of the biggest reasons for the rapid adoption and spread of the web. Even if you forget to close a tag, your page will usually still display.}\par


\vspace{\baselineskip}
\setlength{\parskip}{9.0pt}

\end{itemize}\subsection*{1. Missing or incorrect DOCTYPE.}
\addcontentsline{toc}{subsection}{1. Missing or incorrect DOCTYPE.}
\textcolor[HTML]{0D0D0D}{The DOCTYPE tells Web browsers what version of HTML your page is using. Technically, it refers to a Document Type Definition (DTD) that basically specifies the rules for that version of HTML.\\
The DOCTYPE should always the the very first line of your HTML code and it IS case sensitive.\\
In HTML 4.01 there are three primary DOCTYPE's}\par

\begin{itemize}
	\item \textcolor[HTML]{0D0D0D}{The HTML 4.01 Strict DTD includes all elements and attributes that have not been deprecated or do not appear in frameset documents. For documents that use this DTD, use this document type declaration: <!DOCTYPE HTML PUBLIC "-//W3C//DTD HTML 4.01//EN" "http://www.w3.org/TR/html4/strict.dtd">}\par

	\item \textcolor[HTML]{0D0D0D}{The HTML 4.01 Transitional DTD includes everything in the strict DTD plus deprecated elements and attributes (most of which concern visual presentation). For documents that use this DTD, use this document type declaration: <!DOCTYPE HTML PUBLIC "-//W3C//DTD HTML 4.01 Transitional//EN" "http://www.w3.org/TR/html4/loose.dtd">}\par

	\item \textcolor[HTML]{0D0D0D}{The HTML 4.01 Frameset DTD includes everything in the transitional DTD plus frames as well. For documents that use this DTD, use this document type declaration: <!DOCTYPE HTML PUBLIC "-//W3C//DTD HTML 4.01 Frameset//EN" "http://www.w3.org/TR/html4/frameset.dtd"}
\end{itemize}\par

\begin{adjustwidth}{0.2in}{0.84in}
\subsection*{2. Missing Character Encoding}
\addcontentsline{toc}{subsection}{2. Missing Character Encoding}
\end{adjustwidth}

\begin{adjustwidth}{0.2in}{0.84in}
\textcolor[HTML]{0D0D0D}{All Web pages should define the character set that they are currently using. Though character sets are rather technical, they simply tell the Web browser what set of characters are used in the page.}\par

\end{adjustwidth}

\begin{adjustwidth}{0.2in}{0.84in}
\textcolor[HTML]{0D0D0D}{If a page containing English characters found on typical keyboards will have a different character set than one that should display Japanese characters. The character encoding tells the user agent (browser or assistive device) what kind of data to read and display. For most English Web pages, the character encoding will be entered into the Web page like this: <meta http-equiv="Content-Type" content="text/html; charset=iso-8859-1"> This meta tag should be within the <head> and </head> tags of your Web page and is not case sensitive.}\par

\end{adjustwidth}

\begin{adjustwidth}{0.2in}{0.84in}
\subsection*{3. Unsupported tags or attributes }
\addcontentsline{toc}{subsection}{3. Unsupported tags or attributes }
\end{adjustwidth}

\begin{adjustwidth}{0.2in}{0.84in}
\textcolor[HTML]{0D0D0D}{Use of code that is not part of the HTML standards is not appropriate. These include the <BLINK> and <MARQUEE> tags, among others. There are also many attributes of HTML tags that many browser will recognize, but that are not part of the HTML standard. Commonly used attributes that are improper are attributes in the <body> tags that modify margin size, such as <body marginwidth="0">. These tags and attributes vary based on the version of HTML that you are developing in. For accessibility and compatibility reasons, we should all be using AT LEAST HTML version 4.01. To find out if your page contains unsupported HTML tags or attributes, validate it at the \href{http://validator.w3.org/}{W3C's HTML Validator}. If you don't have a DOCTYPE, then it won't know which version of HTML to validate your page with.}\par

\end{adjustwidth}

\begin{adjustwidth}{0.2in}{0.84in}
\subsection*{4. Improperly formatted HTML}
\addcontentsline{toc}{subsection}{4. Improperly formatted HTML}
\end{adjustwidth}

\begin{adjustwidth}{0.2in}{0.84in}
\textcolor[HTML]{0D0D0D}{The most common mistakes in HTML are usually just plain human mistakes. Here's a list of HTML no-no's:}\par

\end{adjustwidth}

\setlength{\parskip}{5.04pt}
\begin{itemize}
	\item \textcolor[HTML]{0D0D0D}{Missing quotation marks for attribute values.\\
Though older versions of HTML do not require that you surround values with quotations marks, future versions (including XHTML) will. Though you can get away with making this mistake in most browsers, placing quotes around values is suggested.\\
Examples of what NOT to do:\\
<img src=myimage.gif>\\
<font color=$\#$ FF00FF>\\
<p style=font-face: arial, geneva>}\par

	\item \textcolor[HTML]{0D0D0D}{Missing closing tags\\
Most HTML tags have both an opening and closing tag (i.e., <b> and </b>). If a tag mark's up or surrounds any other content, then it must be closed. One exception to this is the <p> tag. XHTML (which we'll talk about later) requires that ALL tags be closed. I recommend closing the <p> tag, even if it is not required now. This usually makes editing your HTML easier as well.}\par

\setlength{\parskip}{12.0pt}
	\item \textcolor[HTML]{0D0D0D}{Improper nesting of HTML tags.\\
HTML tags must be closed in the opposite order than which they were opened. I like to draw imaginary arching lines from matching opening and closing tags. If any of the lines cross, then you have probably nested improperly.}\par



%%%%%%%%%%%%%%%%%%%% Figure/Image No: 1 starts here %%%%%%%%%%%%%%%%%%%%

\begin{figure}[H]
	\begin{Center}
		\includegraphics[width=4.84in,height=0.85in]{./media/image1.gif}
	\end{Center}
\end{figure}


%%%%%%%%%%%%%%%%%%%% Figure/Image No: 1 Ends here %%%%%%%%%%%%%%%%%%%%

\\
\textbf{\textcolor[HTML]{0D0D0D}{Correct}}\textcolor[HTML]{0D0D0D}{: \\
}\par


\vspace{\baselineskip}

%%%%%%%%%%%%%%%%%%%% Figure/Image No: 2 starts here %%%%%%%%%%%%%%%%%%%%

\begin{figure}[H]
	\begin{Center}
		\includegraphics[width=4.6in,height=0.9in]{./media/image2.gif}
	\end{Center}
\end{figure}


%%%%%%%%%%%%%%%%%%%% Figure/Image No: 2 Ends here %%%%%%%%%%%%%%%%%%%%

\begin{justify}
\textbf{\textcolor[HTML]{0D0D0D}{Incorrect:  \\
}}
\end{justify}\par

\setlength{\parskip}{5.04pt}
	\item \textcolor[HTML]{0D0D0D}{It is very common to improperly code when nesting lists (such as those your viewing right now). Any nested <UL> or <OL> must be enclosed within a parent <LI>.}\par

	\item \textcolor[HTML]{0D0D0D}{Using HTML tags for the wrong purpose\\
A common misuse of HTML is using list tags <UL> or <OL> to simulate paragraph indents. HTML tags should only be used for the purpose they were intended. List tags should be used for lists, <BLOCKQUOTE> for long quotes, and so forth.}
\end{itemize}\par


\vspace{\baselineskip}
\vspace{\baselineskip}\setlength{\parskip}{9.0pt}
\begin{adjustwidth}{0.2in}{0.84in}
\subsection*{5. Improper Tables}
\addcontentsline{toc}{subsection}{5. Improper Tables}
\end{adjustwidth}

\begin{adjustwidth}{0.2in}{0.84in}
\textcolor[HTML]{0D0D0D}{Tables are a common culprit of improper HTML. It is easy to incorrectly code tables and most browsers will let you get away with it. Assistive technologies are very strict about proper table structure. Common table mistakes are:}\par

\end{adjustwidth}

\setlength{\parskip}{5.04pt}
\begin{itemize}
	\item \textcolor[HTML]{0D0D0D}{Not closing the <table>, <tr>, or <td> tags or closing them improperly (see above)}\par

	\item \textcolor[HTML]{0D0D0D}{Inserting <td>'s outside of a <tr>}\par

	\item \textcolor[HTML]{0D0D0D}{Creating tables with differing numbers of cells (or rowspan/colspan)in each row}\par

	\item \textcolor[HTML]{0D0D0D}{Placing tables within inline elements, such as <b> or <h1>}\par

	\item \textcolor[HTML]{0D0D0D}{Surrounding table cells or rows with text formatting tags (i.e., <table><b><tr><td>I am bold</td></tr></b></table>)}\par

	\item \textcolor[HTML]{0D0D0D}{Data tables should have a caption, immediately after the opening table tag - <table><caption>Data from Jello Eating Contest</caption><tr> ...}
\end{itemize}\par

\setlength{\parskip}{9.0pt}
\begin{adjustwidth}{0.2in}{0.84in}
\subsection*{6. Missing ALT Text}
\addcontentsline{toc}{subsection}{6. Missing ALT Text}
\end{adjustwidth}

\begin{adjustwidth}{0.2in}{0.84in}
\textcolor[HTML]{0D0D0D}{All images must have the alt attribute: <img src="image.gif" alt="image description">. As of HTML version 4.01, this is required.}\par

\end{adjustwidth}

\begin{adjustwidth}{0.2in}{0.84in}
\subsection*{7. Head content must be within the <head>}
\addcontentsline{toc}{subsection}{7. Head content must be within the <head>}
\end{adjustwidth}

\begin{adjustwidth}{0.2in}{0.84in}
\textcolor[HTML]{0D0D0D}{<title>, <meta>, and <style> tags must be within the <head> and </head> tags.}\par

\end{adjustwidth}

\begin{adjustwidth}{0.2in}{0.84in}
\subsection*{8. Missing </body> or </html> tags}
\addcontentsline{toc}{subsection}{8. Missing </body> or </html> tags}
\end{adjustwidth}


\vspace{\baselineskip}\begin{adjustwidth}{0.2in}{0.84in}
\subsection*{9. Improper use of form tags}
\addcontentsline{toc}{subsection}{9. Improper use of form tags}
\end{adjustwidth}

\begin{adjustwidth}{0.2in}{0.84in}
\textcolor[HTML]{0D0D0D}{The form tag is a block-level tag, meaning that it starts a new section of your page (much like <h1> and <p> do). It is a common mistake to use the form tags to surround smaller sections of your page. To avoid having the form insert a blank line when it begins. This is especially common within tables.\\
Incorrect:}\par

\end{adjustwidth}

\begin{adjustwidth}{0.2in}{0.84in}
\textcolor[HTML]{0D0D0D}{<table><form><tr><td>..... </td></tr></form></table>\\
Correct: <form><table><tr>.... </tr></table></form>}\par

\end{adjustwidth}

\begin{adjustwidth}{0.2in}{0.84in}
\subsection*{10. align=absmiddle}
\addcontentsline{toc}{subsection}{10. align=absmiddle}
\end{adjustwidth}

\begin{adjustwidth}{0.2in}{0.84in}
\textcolor[HTML]{0D0D0D}{This commonly used HTML extension is not proper HTML for the img tag (i.e., <img src="image.gif" align="absmiddle">). This attribute IS supported by the major browsers, but if you want your code to be correct, use either align=middle or CSS to align text to the middle of images.}\par

\end{adjustwidth}


\vspace{\baselineskip}
\vspace{\baselineskip}
\vspace{\baselineskip}\begin{adjustwidth}{0.2in}{0.84in}
\subsection*{11. Missing script type}
\addcontentsline{toc}{subsection}{11. Missing script type}
\end{adjustwidth}

\begin{adjustwidth}{0.2in}{0.84in}
\textcolor[HTML]{0D0D0D}{Scripting languages such as JavaScript and VBScript are becoming very popular. HTML standards require that you identify the type of scripting language that is being used. Most scripts include the language attribute. This alone is not enough, you must also include a type attribute. In fact, in the future, the language attribute will be replaced with the type attribute.\\
<script type="text/javascript">}\par

\end{adjustwidth}

\begin{adjustwidth}{0.2in}{0.84in}
\subsection*{12. Missing <noscript>}
\addcontentsline{toc}{subsection}{12. Missing <noscript>}
\end{adjustwidth}

\begin{adjustwidth}{0.2in}{0.84in}
\textcolor[HTML]{0D0D0D}{Any JavaScript that performs a function or outputs information must have a <noscript> tag that provides an alternative or explanation for what the JavaScript does.}\par

\end{adjustwidth}

\begin{adjustwidth}{0.2in}{0.84in}
\textcolor[HTML]{0D0D0D}{<script type="text/javascript">\\
.... javascript stuff here ...\\
</script>\\
<noscript>\\
<P>Access the <A href="http://someplace.com/data">data.</A>\\
</noscript>}\par

\end{adjustwidth}

\begin{adjustwidth}{0.2in}{0.84in}
\textcolor[HTML]{0D0D0D}{This list could probably go on and on. The best way to determine what mistakes you are making is to \href{http://validator.w3.org/}{\textbf{validate your page}. A common misconception about proper HTML is that if you use a WYSIWYG Web development program to design your page that your HTML will automatically be valid. This is incorrect. Many popular development programs will create improper code or allow you the functionality to create improper code. Always check your HTML for validity to be sure.}}\par

\end{adjustwidth}


\vspace{\baselineskip}

\vspace{\baselineskip}

\vspace{\baselineskip}

\vspace{\baselineskip}

\vspace{\baselineskip}

\vspace{\baselineskip}

\vspace{\baselineskip}

\vspace{\baselineskip}

\vspace{\baselineskip}

\vspace{\baselineskip}

\vspace{\baselineskip}

\vspace{\baselineskip}

\vspace{\baselineskip}
\begin{adjustwidth}{2.7in}{0.84in}
{\fontsize{24pt}{28.8pt}\selectfont \textbf{\textcolor[HTML]{0D0D0D}{LITERATURE REVIEW}}\par}\par

\end{adjustwidth}


\vspace{\baselineskip}

\vspace{\baselineskip}
\vspace{\baselineskip}

\vspace{\baselineskip}
\begin{adjustwidth}{0.2in}{0.84in}
\textcolor[HTML]{0D0D0D}{There are many problems which are faced by students during initial stage of programming\ and teachers also faces many challenges like solving the programming error of all the students.  Because it was\  not\ possible for a single teacher to handle an entire lab so  to increase the student, faculty ratio, M.Tech and P.hD students are also added as T.A.(Teaching Assistant). But this add-on is not enough to give support to all the students, so it is needed to introduce something new which can help or assist all the students with programming simultaneously.}\par

\end{adjustwidth}

\begin{adjustwidth}{0.2in}{0.84in}
\textcolor[HTML]{0D0D0D}{For quite a long time, educators and scientists have been endeavoring to enhance or upgrade the learning procedure of understudies. In this procedure, it is vital to know whether understudies have misinterpretations in their reasonable comprehension or not. The investigation of these components is turning into an important research zone in science and building instruction.}\par

\end{adjustwidth}

\begin{adjustwidth}{0.2in}{0.84in}
\textcolor[HTML]{0D0D0D}{\  }\par

\end{adjustwidth}

\begin{adjustwidth}{0.2in}{0.84in}
\textcolor[HTML]{0D0D0D}{Five categories of specific objectives that can help a teacher to design a unique course or groups of courses are:}\par

\end{adjustwidth}


\vspace{\baselineskip}
	\item \textcolor[HTML]{0D0D0D}{Educational objectives concerned with social frames of mind toward social  change.}\par


\vspace{\baselineskip}
	\item \textcolor[HTML]{0D0D0D}{Instructive goals concerned about the capacity to break down future improvements}\par


\vspace{\baselineskip}
	\item \textcolor[HTML]{0D0D0D}{educational objectives concerned about fundamental and connected critical thinking aptitude. Instructive goals concerned with analytical and synthesizing skills. 5.Instructive goals concerned about qualities contemplation. }\par


\vspace{\baselineskip}

\vspace{\baselineskip}
\begin{adjustwidth}{0.1in}{0.84in}
\textcolor[HTML]{0D0D0D}{\ \ \ \ \ \ \ \ \ \ \ \ \ \ \ \ \ \ \ \ \ \ \ \ \ \ \ \  Everybody should figure out how to be a powerful supporter of participatory procedures, many should figure out how to lead them, and some should figure out how to structure the stages on which they run. Exploiting these chances and staying away from potential dangers will require creating office with innovation interceded social cooperation.}\par

\end{adjustwidth}


\vspace{\baselineskip}
\vspace{\baselineskip}
\vspace{\baselineskip}
\vspace{\baselineskip}
\vspace{\baselineskip}
\vspace{\baselineskip}\begin{adjustwidth}{0.2in}{0.84in}
\textcolor[HTML]{0D0D0D}{Various endeavors try to expand mindfulness, intrigue, and support in logical and innovative fields at the precollege level. Studies have demonstrated these understudies are at a basic age where introduction to building and other related fields, for example, science, arithmetic, and innovation significantly affect their profession  objective.}\par

\end{adjustwidth}


\vspace{\baselineskip}\begin{adjustwidth}{0.2in}{0.84in}
\begin{justify}
{\fontsize{13pt}{15.6pt}\selectfont \textbf{\textcolor[HTML]{0D0D0D}{3.1 Summary of Reviewed Literature}}\par}
\end{justify}\par

\end{adjustwidth}


\vspace{\baselineskip}
\begin{adjustwidth}{0.2in}{0.84in}
\textcolor[HTML]{0D0D0D}{A lot work has already been done in the field of education but still lots of work is still remaining to do. Advancement is needed in every field either it is learning or delivering knowledge. So by studying many thing,it has been concluded that an automatic tool is needed for correcting syntax error to save time of students and teachers.}\par

\end{adjustwidth}


\vspace{\baselineskip}
\vspace{\baselineskip}

\vspace{\baselineskip}
\begin{adjustwidth}{4.67in}{0.0in}
{\fontsize{24pt}{28.8pt}\selectfont \textbf{\textcolor[HTML]{0D0D0D}{ANALYSIS}}\par}\par

\end{adjustwidth}


\vspace{\baselineskip}

\vspace{\baselineskip}
\vspace{\baselineskip}

\vspace{\baselineskip}
\begin{adjustwidth}{0.2in}{0.84in}
\begin{justify}
{\fontsize{16pt}{19.2pt}\selectfont \textbf{\textcolor[HTML]{0D0D0D}{4.1 Detailed problem statement}}\par}
\end{justify}\par

\end{adjustwidth}


\vspace{\baselineskip}
\begin{adjustwidth}{0.2in}{0.84in}
\begin{justify}
{\fontsize{14pt}{16.8pt}\selectfont \textbf{\textcolor[HTML]{0D0D0D}{Indent}}\par}
\end{justify}\par

\end{adjustwidth}

\setlength{\parskip}{16.8pt}
\begin{adjustwidth}{0.2in}{0.84in}
\begin{justify}
\textcolor[HTML]{0D0D0D}{The every issue of stationery should be based on requisition. The departments can prepare an indent whenever there is a need of stationery. The departmental heads should sign the indent. The storekeeper can issue the stationery after receiving indent. No issue will be made without indent.}
\end{justify}\par

\end{adjustwidth}


\vspace{\baselineskip}
\begin{adjustwidth}{0.2in}{0.84in}
\subsubsection*{Delivery}
\addcontentsline{toc}{subsubsection}{Delivery}
\end{adjustwidth}

\begin{adjustwidth}{0.2in}{0.84in}
\textcolor[HTML]{0D0D0D}{The storekeepers can deliver stationery at the work place of every department. There is no need of deputing any person to get the stationery items from the stores by the departments. This saves time, labour and pilferage.}\par

\end{adjustwidth}


\vspace{\baselineskip}
\begin{adjustwidth}{0.2in}{0.84in}
\subsubsection*{Issue Quantity}
\addcontentsline{toc}{subsubsection}{Issue Quantity}
\end{adjustwidth}

\begin{adjustwidth}{0.2in}{0.84in}
\textcolor[HTML]{0D0D0D}{Every stationery is to be issued in a specified quantity. There must be a unit of measurement for stationery. The paper may be issued in ream, pens, pencils, cutters, erasers and the like are issue in dozens. Gum is issued in number. This unit of measurement facilitates stock control and helpful to both stores and departments}\par

\end{adjustwidth}


\vspace{\baselineskip}
\begin{adjustwidth}{0.2in}{0.84in}
\subsubsection*{Accounting}
\addcontentsline{toc}{subsubsection}{Accounting}
\end{adjustwidth}

\begin{adjustwidth}{0.2in}{0.84in}
\textcolor[HTML]{0D0D0D}{Both stores and department should prepare the accounts for stationery. The storekeeper maintains the database. The database is to be updated with every issue of stationery. The physical stock should tally with the database. Likewise, each department should maintain stationery record, make entries for receipt, issue, and balance in the database to keep the stationery stock up to date. Database should contains the details like date, indent number, indenting department, quantity issued etc. Proper entries should be made in Issue Register for effective control system.}\par

\end{adjustwidth}


\vspace{\baselineskip}
\begin{adjustwidth}{0.2in}{0.84in}
\begin{justify}
{\fontsize{14pt}{16.8pt}\selectfont \textbf{\textcolor[HTML]{0D0D0D}{Stock Maintenance}}\par}
\end{justify}\par

\end{adjustwidth}

\begin{adjustwidth}{0.2in}{0.84in}
\begin{justify}
\textcolor[HTML]{0D0D0D}{Stock levels for each stationery items are fixed and maintained. The stationery items should be purchased at regular intervals in order to maintain required stock levels. Required stock levels ensure continuous supply of stationery. This should be a continuous activity. The department should not be allowed to receive the stationery without indent. Drawings of the stationery will be honored depending upon the honesty and sincerity of the person drawings the stationery.}
\end{justify}\par

\end{adjustwidth}


\vspace{\baselineskip}

\vspace{\baselineskip}

\vspace{\baselineskip}

\vspace{\baselineskip}

\vspace{\baselineskip}
\begin{adjustwidth}{0.2in}{0.84in}
\begin{justify}
{\fontsize{14pt}{16.8pt}\selectfont \textbf{\textcolor[HTML]{0D0D0D}{4.2\ \  Usecase}}\par}
\end{justify}\par

\end{adjustwidth}

\begin{adjustwidth}{0.2in}{0.84in}
\begin{justify}
\textcolor[HTML]{0D0D0D}{A use case is a written description of how users will perform tasks on your website.  It outlines, from a user’s point of view, a system’s behavior as it responds to a request. Each use case is represented as a sequence of simple steps, beginning with a user's goal and ending when that goal is fulfilled.}
\end{justify}\par

\end{adjustwidth}


\vspace{\baselineskip}
\begin{adjustwidth}{0.2in}{0.84in}
\subsection*{Elements of a Use Case}
\addcontentsline{toc}{subsection}{Elements of a Use Case}
\end{adjustwidth}

\setlength{\parskip}{9.6pt}
\begin{adjustwidth}{0.2in}{0.84in}
\textcolor[HTML]{0D0D0D}{Depending on how in depth and complex you want or need to get, use cases describe a combination of the following elements:}\par

\end{adjustwidth}

\begin{itemize}
	\item \textcolor[HTML]{0D0D0D}{Actor – anyone or anything that performs a behavior (who is using the system)}\par

	\item \textcolor[HTML]{0D0D0D}{Stakeholder – someone or something with vested interests in the behavior of the system under discussion (SUD)}\par

	\item \textcolor[HTML]{0D0D0D}{Primary Actor – stakeholder who initiates an interaction with the system to achieve a goal}\par

	\item \textcolor[HTML]{0D0D0D}{Preconditions – what must be true or happen before and after the use case runs.}\par

	\item \textcolor[HTML]{0D0D0D}{Triggers – this is the event that causes the use case to be initiated.}\par

	\item \textcolor[HTML]{0D0D0D}{Main success scenarios [Basic Flow] – use case in which nothing goes wrong.}\par

	\item \textcolor[HTML]{0D0D0D}{Alternative paths [Alternative Flow] – these paths are a variation on the main theme. }
\end{itemize}\par


\vspace{\baselineskip}
\begin{adjustwidth}{0.2in}{0.84in}
\begin{justify}
\textcolor[HTML]{0D0D0D}{These exceptions are what happen when things go wrong at the system level.}
\end{justify}\par

\end{adjustwidth}


\vspace{\baselineskip}

\vspace{\baselineskip}


%%%%%%%%%%%%%%%%%%%% Figure/Image No: 3 starts here %%%%%%%%%%%%%%%%%%%%

\begin{figure}[H]
	\begin{Center}
		\includegraphics[width=6.36in,height=4.01in]{./media/image3.png}
	\end{Center}
\end{figure}


%%%%%%%%%%%%%%%%%%%% Figure/Image No: 3 Ends here %%%%%%%%%%%%%%%%%%%%

\par


\vspace{\baselineskip}

\vspace{\baselineskip}

\vspace{\baselineskip}

\vspace{\baselineskip}

\vspace{\baselineskip}

\vspace{\baselineskip}

\vspace{\baselineskip}
\begin{adjustwidth}{0.2in}{0.84in}
\begin{justify}
{\fontsize{14pt}{16.8pt}\selectfont \textbf{\textcolor[HTML]{0D0D0D}{4.3 Activity Diagram}}\par}
\end{justify}\par

\end{adjustwidth}


\vspace{\baselineskip}
\begin{adjustwidth}{0.2in}{0.84in}
\begin{justify}
\textcolor[HTML]{0D0D0D}{Activity diagram is defined as a UML diagram that focuses on the execution and flow of the behavior of a system instead of implementation. It is also called object-oriented flowchart. Activity diagrams consist of activities that are made up of actions which apply to behavioral modeling technology.}
\end{justify}\par

\end{adjustwidth}


\vspace{\baselineskip}
\begin{adjustwidth}{0.2in}{0.84in}
\subsection*{Components of Activity Diagram}
\addcontentsline{toc}{subsection}{Components of Activity Diagram}
\end{adjustwidth}


\vspace{\baselineskip}
\begin{adjustwidth}{0.2in}{0.84in}
\subsubsection*{Activities}
\addcontentsline{toc}{subsubsection}{Activities}
\end{adjustwidth}

\begin{adjustwidth}{0.2in}{0.84in}
\textcolor[HTML]{0D0D0D}{It is a behavior that is divided into one or more actions. Activities are a network of nodes connected by edges. There can be action nodes, control nodes, or object nodes. Action nodes represent some action. Control nodes represent the control flow of an activity. Object nodes are used to describe objects used inside an activity. Edges are used to show a path or a flow of execution. Activities start at an initial node and terminate at a final node.}\par

\end{adjustwidth}


\vspace{\baselineskip}
\begin{adjustwidth}{0.2in}{0.84in}
\subsubsection*{Activity partition/swimlane}
\addcontentsline{toc}{subsubsection}{Activity partition/swimlane}
\end{adjustwidth}

\begin{adjustwidth}{0.2in}{0.84in}
\textcolor[HTML]{0D0D0D}{An activity partition or a swimlane is a high-level grouping of a set of related actions. A single partition can refer to many things, such as classes, use cases, components, or interfaces.}\par

\end{adjustwidth}

\begin{adjustwidth}{0.2in}{0.84in}
\textcolor[HTML]{0D0D0D}{If a partition cannot be shown clearly, then the name of a partition is written on top of the name of an activity.}\par

\end{adjustwidth}

\begin{adjustwidth}{0.2in}{0.84in}
\subsubsection*{Fork and Join nodes}
\addcontentsline{toc}{subsubsection}{Fork and Join nodes}
\end{adjustwidth}

\begin{adjustwidth}{0.2in}{0.84in}
\textcolor[HTML]{0D0D0D}{Using a fork and join nodes, concurrent flows within an activity can be generated. A fork node has one incoming edge and numerous outgoing edges. It is similar to one too many decision parameters. When data arrives at an incoming edge, it is duplicated and split across numerous outgoing edges simultaneously. A single incoming flow is divided into multiple parallel flows.}\par

\end{adjustwidth}

\begin{adjustwidth}{0.2in}{0.84in}
\textcolor[HTML]{0D0D0D}{A join node is opposite of a fork node as It has many incoming edges and a single outgoing edge. It performs logical AND operation on all the incoming edges. This helps you to synchronize the input flow across a single output edge.}\par

\end{adjustwidth}

\begin{adjustwidth}{0.2in}{0.84in}
\textcolor[HTML]{0D0D0D}{Pins}\par

\end{adjustwidth}

\begin{adjustwidth}{0.2in}{0.84in}
\textcolor[HTML]{0D0D0D}{An activity diagram that has a lot of flows gets very complicated and messy.}\par

\end{adjustwidth}

\begin{adjustwidth}{0.2in}{0.84in}
\textcolor[HTML]{0D0D0D}{Pins are used to clearing up the things. It provides a way to manage the execution flow of activity by sorting all the flows and cleaning up messy thins. It is an object node that represents one input to or an output from an action.}\par

\end{adjustwidth}

\begin{adjustwidth}{0.2in}{0.84in}
\textcolor[HTML]{0D0D0D}{Both input and output pins have precisely one edge.}\par

\end{adjustwidth}


\vspace{\baselineskip}


%%%%%%%%%%%%%%%%%%%% Figure/Image No: 4 starts here %%%%%%%%%%%%%%%%%%%%

\begin{figure}[H]
	\begin{Center}
		\includegraphics[width=5.43in,height=6.69in]{./media/image4.png}
	\end{Center}
\end{figure}


%%%%%%%%%%%%%%%%%%%% Figure/Image No: 4 Ends here %%%%%%%%%%%%%%%%%%%%

\par


\vspace{\baselineskip}

\vspace{\baselineskip}
\begin{adjustwidth}{0.2in}{0.84in}
\begin{justify}
\textbf{\textcolor[HTML]{0D0D0D}{4.4 Functional Requirement}}
\end{justify}\par

\end{adjustwidth}


\vspace{\baselineskip}
\begin{adjustwidth}{0.2in}{0.84in}
\begin{justify}
\textcolor[HTML]{0D0D0D}{A Functional Requirement (FR) is a description of the service that the software must offer. It describes a software system or its component. A function is nothing but inputs to the software system, its behavior, and outputs. It can be a calculation, data manipulation, business process, user interaction, or any other specific functionality which defines what function a system is likely to perform. Functional Requirements are also called Functional}
\end{justify}\par

\end{adjustwidth}


\vspace{\baselineskip}
\begin{adjustwidth}{0.2in}{0.84in}
\begin{justify}
\textcolor[HTML]{0D0D0D}{Requisite:- A person must be able to fill requisition form in order to request for the stationary.}
\end{justify}\par

\end{adjustwidth}


\vspace{\baselineskip}
\begin{adjustwidth}{0.2in}{0.84in}
\begin{justify}
\textcolor[HTML]{0D0D0D}{Stock\ Update:-  A store manager must be able to update stock information in database after each requisition .}
\end{justify}\par

\end{adjustwidth}


\vspace{\baselineskip}
\begin{adjustwidth}{0.2in}{0.84in}
\begin{justify}
\textcolor[HTML]{0D0D0D}{Stock Request:- Faculty should be able to request for required stationary to the store manager.}
\end{justify}\par

\end{adjustwidth}


\vspace{\baselineskip}

\vspace{\baselineskip}

\vspace{\baselineskip}

\vspace{\baselineskip}
\begin{adjustwidth}{0.2in}{0.84in}
\begin{justify}
\textcolor[HTML]{0D0D0D}{Print Receipt :- The store manager must be able to print the receipt of supplied stationary materials.\  }
\end{justify}\par

\end{adjustwidth}


\vspace{\baselineskip}
\begin{adjustwidth}{0.2in}{0.84in}
\begin{justify}
\textbf{\textcolor[HTML]{0D0D0D}{View Stock:- }}\textcolor[HTML]{0D0D0D}{The Faculty and Store Manager both must be able to view the available stock.}
\end{justify}\par

\end{adjustwidth}


\vspace{\baselineskip}

\vspace{\baselineskip}
\begin{adjustwidth}{0.2in}{0.84in}
\begin{justify}
\textbf{\textcolor[HTML]{0D0D0D}{4.5\  Non-Functional Requirement}}
\end{justify}\par

\end{adjustwidth}


\vspace{\baselineskip}
\begin{adjustwidth}{0.2in}{0.84in}
\begin{justify}
\textcolor[HTML]{0D0D0D}{The non-functional requirements deals with the quality of the systems needed to be developed from different evaluation point of view like the response time of the system to a given user queries, the user friendliness of the website and These requirements do not directly affect the performance of the system but they are important.}
\end{justify}\par

\end{adjustwidth}

\begin{adjustwidth}{0.2in}{0.84in}
\begin{justify}
\textcolor[HTML]{0D0D0D}{ }
\end{justify}\par

\end{adjustwidth}

\begin{adjustwidth}{0.2in}{0.84in}
\begin{justify}
\textcolor[HTML]{0D0D0D}{ }
\end{justify}\par

\end{adjustwidth}

\begin{adjustwidth}{0.2in}{0.84in}
\begin{justify}
\textbf{\textcolor[HTML]{0D0D0D}{Maintenance:-}}\textcolor[HTML]{0D0D0D}{ The Stationary Management System is being developed in java it is an object oriented programming language and shall be easy to maintain}
\end{justify}\par

\end{adjustwidth}

\begin{adjustwidth}{0.2in}{0.84in}
\begin{justify}
\textcolor[HTML]{0D0D0D}{ }
\end{justify}\par

\end{adjustwidth}

\begin{adjustwidth}{0.2in}{0.84in}
\begin{justify}
\textbf{\textcolor[HTML]{0D0D0D}{Portability:-}}\textcolor[HTML]{0D0D0D}{The Stationary Management System shall run in any Microsoft Windows environment and Linux environment.}
\end{justify}\par

\end{adjustwidth}

\begin{adjustwidth}{0.2in}{0.84in}
\begin{justify}
\textcolor[HTML]{0D0D0D}{Reliability: - The Stationary Management System service should not access without authenticate user.}
\end{justify}\par

\end{adjustwidth}

\begin{adjustwidth}{0.2in}{0.84in}
\begin{justify}
\textcolor[HTML]{0D0D0D}{ }
\end{justify}\par

\end{adjustwidth}

\begin{adjustwidth}{0.2in}{0.84in}
\begin{justify}
\textbf{\textcolor[HTML]{0D0D0D}{Standards Compliance: -}}\textcolor[HTML]{0D0D0D}{ The graphical user interface of the system shall have easily understood to the user (have consistent look and feel graphical user interface).}
\end{justify}\par

\end{adjustwidth}

\begin{adjustwidth}{0.2in}{0.84in}
\begin{justify}
\textcolor[HTML]{0D0D0D}{ }
\end{justify}\par

\end{adjustwidth}

\begin{adjustwidth}{0.2in}{0.84in}
\begin{justify}
\textbf{\textcolor[HTML]{0D0D0D}{Performance:}}\textcolor[HTML]{0D0D0D}{ -Acceptable response times for system functionality.}
\end{justify}\par

\end{adjustwidth}

\begin{adjustwidth}{0.2in}{0.84in}
\begin{justify}
\textcolor[HTML]{0D0D0D}{ }
\end{justify}\par

\end{adjustwidth}

\begin{adjustwidth}{0.2in}{0.84in}
\begin{justify}
\textbf{\textcolor[HTML]{0D0D0D}{Security: -}}\textcolor[HTML]{0D0D0D}{ The Stationary Management System should be secure and no unwanted access should be allowed.}
\end{justify}\par

\end{adjustwidth}


\vspace{\baselineskip}

\vspace{\baselineskip}
\begin{adjustwidth}{0.2in}{0.84in}
{\fontsize{14pt}{16.8pt}\selectfont \textbf{\textcolor[HTML]{0D0D0D}{4.6 System Requirement}}\par}\par

\end{adjustwidth}

\setlength{\parskip}{5.04pt}
\begin{itemize}
	\item \textcolor[HTML]{0D0D0D}{CPU: Pentium 4 processor or higher}\par

	\item \textcolor[HTML]{0D0D0D}{CPU SPEED: 1.7 GHz}\par

	\item \textcolor[HTML]{0D0D0D}{RAM: 512 MB}\par

	\item \textcolor[HTML]{0D0D0D}{OS: Windows 7 /Vista/XP or higher}\par

	\item \textcolor[HTML]{0D0D0D}{FREE DISK SPACE: 1 GB}
\end{itemize}\par


\vspace{\baselineskip}

\vspace{\baselineskip}

\vspace{\baselineskip}

\vspace{\baselineskip}

\vspace{\baselineskip}

\vspace{\baselineskip}

\vspace{\baselineskip}

\vspace{\baselineskip}

\vspace{\baselineskip}
\begin{adjustwidth}{5.0in}{0.84in}
\begin{justify}
{\fontsize{24pt}{28.8pt}\selectfont \textbf{\textcolor[HTML]{0D0D0D}{\ \ \ \  DESIGN}}\par}
\end{justify}\par

\end{adjustwidth}

\tab 
\vspace{\baselineskip}
\vspace{\baselineskip}
\vspace{\baselineskip}
\begin{adjustwidth}{0.3in}{0.84in}
{\fontsize{14pt}{16.8pt}\selectfont \textbf{\textcolor[HTML]{0D0D0D}{\ \ \  }}\par}\par

\end{adjustwidth}

\begin{adjustwidth}{0.3in}{0.84in}
{\fontsize{14pt}{16.8pt}\selectfont \textbf{\textcolor[HTML]{0D0D0D}{5.1 Class Diagram Purpose}}\par}\par

\end{adjustwidth}

\tab 
\vspace{\baselineskip}\begin{adjustwidth}{0.3in}{0.84in}
\textcolor[HTML]{0D0D0D}{Purpose of Class Diagram The purpose of class diagram is to model the static view of an application. Class diagrams are the only diagrams which can be directly mapped with object-oriented languages and thus widely used at the time of construction. UML diagrams like activity diagram, sequence diagram can only give the sequence flow of the application, however class diagram is a bit different. It is the most popular UML diagram in the coder community. The purpose of the class diagram can be summarized as – }\par

\end{adjustwidth}

\begin{adjustwidth}{0.3in}{0.84in}
\textcolor[HTML]{0D0D0D}{$\bullet$  \tab Analysis and design of the static view of an application. }\par

\end{adjustwidth}

\begin{adjustwidth}{0.3in}{0.84in}
\textcolor[HTML]{0D0D0D}{$\bullet$  \tab Describe responsibilities of a system. }\par

\end{adjustwidth}

\begin{adjustwidth}{0.3in}{0.84in}
\textcolor[HTML]{0D0D0D}{$\bullet$  \tab Base for component and deployment diagrams. }\par

\end{adjustwidth}

\begin{adjustwidth}{0.3in}{0.84in}
\textcolor[HTML]{0D0D0D}{$\bullet$  \tab Forward and reverse engineering.}\par

\end{adjustwidth}


\vspace{\baselineskip}

\vspace{\baselineskip}

\vspace{\baselineskip}

\vspace{\baselineskip}
\begin{adjustwidth}{0.3in}{0.84in}
\subsection*{5.2 Class Diagram Description}
\addcontentsline{toc}{subsection}{5.2 Class Diagram Description}
\end{adjustwidth}


\vspace{\baselineskip}
\begin{adjustwidth}{0.3in}{0.84in}
\textbf{\textcolor[HTML]{0D0D0D}{ }}\par

\end{adjustwidth}

\begin{adjustwidth}{0.3in}{0.84in}
\textbf{\textcolor[HTML]{0D0D0D}{Store\ Manager:  Who will manage everything in the system. And Requisite the items via requisition object.}}\par

\end{adjustwidth}


\vspace{\baselineskip}
\begin{adjustwidth}{0.3in}{0.84in}
\textbf{\textcolor[HTML]{0D0D0D}{Requisition: All the requisition data will in arranged.}}\par

\end{adjustwidth}


\vspace{\baselineskip}
\begin{adjustwidth}{0.3in}{0.84in}
\textbf{\textcolor[HTML]{0D0D0D}{Faculty : Faculties will request for stationery and which will be recorded for the track.}}\par

\end{adjustwidth}


\vspace{\baselineskip}
\begin{adjustwidth}{0.3in}{0.84in}
\textbf{\textcolor[HTML]{0D0D0D}{Stock : All the available stock of each item.}}\par

\end{adjustwidth}


\vspace{\baselineskip}

\vspace{\baselineskip}

\vspace{\baselineskip}

\vspace{\baselineskip}

\vspace{\baselineskip}

\vspace{\baselineskip}

\vspace{\baselineskip}

\vspace{\baselineskip}

\vspace{\baselineskip}

\vspace{\baselineskip}

\vspace{\baselineskip}

\vspace{\baselineskip}

\vspace{\baselineskip}

\vspace{\baselineskip}
\begin{adjustwidth}{0.3in}{0.84in}
\subsection*{5.2 Class Diagram}
\addcontentsline{toc}{subsection}{5.2 Class Diagram}
\end{adjustwidth}


\vspace{\baselineskip}

\vspace{\baselineskip}


%%%%%%%%%%%%%%%%%%%% Figure/Image No: 5 starts here %%%%%%%%%%%%%%%%%%%%

\begin{figure}[H]
	\begin{Center}
		\includegraphics[width=5.67in,height=5.32in]{./media/image5.jpeg}
	\end{Center}
\end{figure}


%%%%%%%%%%%%%%%%%%%% Figure/Image No: 5 Ends here %%%%%%%%%%%%%%%%%%%%

\par


\vspace{\baselineskip}
\begin{adjustwidth}{0.3in}{0.84in}
\textcolor[HTML]{0D0D0D}{Figure 5.1: Class diagram.}\par

\end{adjustwidth}


\vspace{\baselineskip}
\vspace{\baselineskip}
\begin{adjustwidth}{0.3in}{0.84in}
{\fontsize{13pt}{15.6pt}\selectfont \textbf{\textcolor[HTML]{0D0D0D}{5.4 Purpose of Sequence Diagram\par} }}\par

\end{adjustwidth}


\vspace{\baselineskip}
\begin{adjustwidth}{0.3in}{0.84in}
\textcolor[HTML]{0D0D0D}{The purpose of interaction diagrams is to visualize the interactive behavior of the system. Visualizing the interaction is a difficult task. Hence, the solution is to use different types of models to capture the different aspects of the interaction. Sequence and collaboration diagrams are used to capture the dynamic nature but from a different angle. The purpose of interaction diagram is – }\par

\end{adjustwidth}

\begin{adjustwidth}{0.3in}{0.84in}
\textcolor[HTML]{0D0D0D}{$\bullet$  To capture the dynamic behaviour of a system. }\par

\end{adjustwidth}

\begin{adjustwidth}{0.3in}{0.84in}
\textcolor[HTML]{0D0D0D}{$\bullet$  To describe the message flow in the system. }\par

\end{adjustwidth}

\begin{adjustwidth}{0.3in}{0.84in}
\textcolor[HTML]{0D0D0D}{$\bullet$  To describe the structural organization of the objects. }\par

\end{adjustwidth}

\begin{adjustwidth}{0.3in}{0.84in}
\textcolor[HTML]{0D0D0D}{$\bullet$  To describe the interaction among objects. }\par

\end{adjustwidth}


\vspace{\baselineskip}

\vspace{\baselineskip}

\vspace{\baselineskip}
\begin{adjustwidth}{0.3in}{0.84in}
{\fontsize{14pt}{16.8pt}\selectfont \textbf{\textcolor[HTML]{0D0D0D}{5.5 Sequence Diagram Description}}\par}\par

\end{adjustwidth}


\vspace{\baselineskip}
\begin{adjustwidth}{0.3in}{0.84in}
\begin{justify}
\textcolor[HTML]{0D0D0D}{It shows the timeline of events happening throughout the whole process.}
\end{justify}\par

\end{adjustwidth}

\begin{adjustwidth}{0.3in}{0.84in}
\begin{justify}
\textcolor[HTML]{0D0D0D}{Chronological explanation of steps is as follow-}
\end{justify}\par

\end{adjustwidth}


\vspace{\baselineskip}
\begin{itemize}
	\item \textcolor[HTML]{0D0D0D}{The Stationary Manager will be have the items and there quantity being requested to the Store so he will then create a Requisition Reciept after printing and getting back all the formalities.}\par


\vspace{\baselineskip}
	\item \textcolor[HTML]{0D0D0D}{The Items are to be filled into the stock so that i can be updated because it might be possible that some of the items are not available at the store the stock will be filled back with the values and The available stock will get updated.}\par


\vspace{\baselineskip}
	\item \textcolor[HTML]{0D0D0D}{Then whenever a faculty need an Item he can basically request for one if available after checking the stocks and the stock will be updated deducting the values delivered.}
\end{itemize}\par


\vspace{\baselineskip}

\vspace{\baselineskip}
\begin{adjustwidth}{0.3in}{0.84in}
\begin{justify}
{\fontsize{14pt}{16.8pt}\selectfont \textbf{\textcolor[HTML]{0D0D0D}{5.6 Sequence Diagram}}\par}
\end{justify}\par

\end{adjustwidth}


\vspace{\baselineskip}

\vspace{\baselineskip}


%%%%%%%%%%%%%%%%%%%% Figure/Image No: 6 starts here %%%%%%%%%%%%%%%%%%%%

\begin{figure}[H]
	\begin{Center}
		\includegraphics[width=6.27in,height=4.78in]{./media/image6.jpeg}
	\end{Center}
\end{figure}


%%%%%%%%%%%%%%%%%%%% Figure/Image No: 6 Ends here %%%%%%%%%%%%%%%%%%%%

\par


\vspace{\baselineskip}

\vspace{\baselineskip}

\vspace{\baselineskip}

\vspace{\baselineskip}
\begin{adjustwidth}{0.3in}{0.84in}
{\fontsize{20pt}{24.0pt}\selectfont \textbf{\textcolor[HTML]{0D0D0D}{ \tab \tab \tab \tab \tab \tab \tab \tab  }}{\fontsize{24pt}{28.8pt}\selectfont \textbf{\textcolor[HTML]{0D0D0D}{Implementation}}\par}\par}\par

\end{adjustwidth}


\vspace{\baselineskip}

\vspace{\baselineskip}
\vspace{\baselineskip}

\vspace{\baselineskip}
\subsection{HTML and CSS}

\vspace{\baselineskip}
\begin{adjustwidth}{0.3in}{0.84in}
\textcolor[HTML]{0D0D0D}{In our work of Web Application we use HTML to create the basic of the web pages. It is used to provide the structure to the web pages. And to make these web pages more attractive we have used CSS. }\par

\end{adjustwidth}


\vspace{\baselineskip}
\begin{adjustwidth}{0.3in}{0.84in}
\textbf{\textcolor[HTML]{0D0D0D}{Commonly used tags of the HTML are:}}\par

\end{adjustwidth}

\begin{adjustwidth}{0.3in}{0.84in}
\textbf{\textcolor[HTML]{0D0D0D}{<html>\tab \tab \tab Defines an HTML tag.}}\par

\end{adjustwidth}

\begin{adjustwidth}{0.3in}{0.84in}
\textbf{\textcolor[HTML]{0D0D0D}{<head>\tab \tab \tab Defines information about the document}}\par

\end{adjustwidth}

\begin{adjustwidth}{0.3in}{0.84in}
\textbf{\textcolor[HTML]{0D0D0D}{<title>\tab \tab \tab Defines title for the document.}}\par

\end{adjustwidth}

\begin{adjustwidth}{0.3in}{0.84in}
\textbf{\textcolor[HTML]{0D0D0D}{<body>\tab \tab \tab Defines the documents body}}\par

\end{adjustwidth}

\begin{adjustwidth}{0.3in}{0.84in}
\textbf{\textcolor[HTML]{0D0D0D}{<h1> to <h6> \tab \tab Defines HTML headings.}}\par

\end{adjustwidth}

\begin{adjustwidth}{0.3in}{0.84in}
\textbf{\textcolor[HTML]{0D0D0D}{<p>\tab \tab \tab Defines a paragraph}}\par

\end{adjustwidth}

\begin{adjustwidth}{0.3in}{0.84in}
\textbf{\textcolor[HTML]{0D0D0D}{<br>\tab \tab \tab Inserts a break line}}\par

\end{adjustwidth}

\begin{adjustwidth}{0.3in}{0.84in}
\textbf{\textcolor[HTML]{0D0D0D}{<form>\tab \tab \tab Defines a form with a submit button}}\par

\end{adjustwidth}

\begin{adjustwidth}{0.3in}{0.84in}
\textbf{\textcolor[HTML]{0D0D0D}{ <input>\tab \tab \tab To take input from the user.}}\par

\end{adjustwidth}

\begin{adjustwidth}{1.4in}{0.84in}
\textcolor[HTML]{0D0D0D}{And many more.}\par

\end{adjustwidth}


\vspace{\baselineskip}
\begin{adjustwidth}{0.3in}{0.84in}
\textbf{\textcolor[HTML]{0D0D0D}{Commonly used tags in CSS:}}\par

\end{adjustwidth}

\begin{adjustwidth}{0.3in}{0.84in}
\textcolor[HTML]{0D0D0D}{<font>\tab \tab To have the different text format}\par

\end{adjustwidth}

\begin{adjustwidth}{0.3in}{0.84in}
\textcolor[HTML]{0D0D0D}{<center>\tab \tab centrally align the text, table etc.}\par

\end{adjustwidth}

\begin{adjustwidth}{1.4in}{0.84in}
\textcolor[HTML]{0D0D0D}{And many more.}\par

\end{adjustwidth}


\vspace{\baselineskip}
\subsection{JavaScript }
\setlength{\parskip}{14.4pt}
\begin{adjustwidth}{0.3in}{0.84in}
\textcolor[HTML]{0D0D0D}{JavaScript is the programming language of HTML and the web page.}\par

\end{adjustwidth}

\begin{adjustwidth}{0.3in}{0.84in}
\textcolor[HTML]{0D0D0D}{JavaScript syntax is the set of rules, how JavaScript programs are constructed:}\par

\end{adjustwidth}

\begin{adjustwidth}{0.3in}{0.84in}
\textcolor[HTML]{0D0D0D}{var x, y, z;       \tab // How to declare variables\\
x = 5; y = 6;     \tab // How to assign values\\
z = x + y;         \tab // How to compute values}\par

\end{adjustwidth}


\vspace{\baselineskip}
\begin{adjustwidth}{0.3in}{0.84in}
\textcolor[HTML]{0D0D0D}{A computer program is a list of "instructions" to be "executed" by a computer.}\par

\end{adjustwidth}

\begin{adjustwidth}{0.3in}{0.84in}
\textcolor[HTML]{0D0D0D}{In a programming language, these programming instructions are called statements.}\par

\end{adjustwidth}

\begin{adjustwidth}{0.3in}{0.84in}
\textcolor[HTML]{0D0D0D}{A JavaScript program is a list of programming statements.}\par

\end{adjustwidth}


\vspace{\baselineskip}
\begin{adjustwidth}{0.3in}{0.84in}
{\fontsize{14pt}{16.8pt}\selectfont \textbf{\textcolor[HTML]{0D0D0D}{JSON}}\par}\par

\end{adjustwidth}

\setlength{\parskip}{5.04pt}
\begin{itemize}
	\item \textcolor[HTML]{0D0D0D}{JSON stands for \textbf{J}ava\textbf{S}cript \textbf{O}bject \textbf{N}otation}\par

	\item \textcolor[HTML]{0D0D0D}{JSON is a lightweight data interchange format}\par

	\item \textcolor[HTML]{0D0D0D}{JSON is language independent \textbf{$\ast$ }}\par

	\item \textcolor[HTML]{0D0D0D}{JSON is "self-describing" and easy to understand}
\end{itemize}\par


\vspace{\baselineskip}
\begin{adjustwidth}{0.3in}{0.84in}
\subsection*{Servlet}
\addcontentsline{toc}{subsection}{Servlet}
\end{adjustwidth}

\begin{adjustwidth}{0.3in}{0.84in}
\textcolor[HTML]{0D0D0D}{Servlet technology is used to create a web application (resides at server side and generates a dynamic web page).}\par

\end{adjustwidth}

\begin{adjustwidth}{0.3in}{0.84in}
\textcolor[HTML]{0D0D0D}{Servlet technology is robust and scalable because of java language. Before Servlet, CGI (Common Gateway Interface) scripting language was common as a server-side programming language. However, there were many disadvantages to this technology. We have discussed these disadvantages below.}\par

\end{adjustwidth}

\begin{adjustwidth}{0.3in}{0.84in}
\textcolor[HTML]{0D0D0D}{There are many interfaces and classes in the Servlet API such as Servlet, GenericServlet, HttpServlet, ServletRequest, ServletResponse, etc.}\par

\end{adjustwidth}

\begin{adjustwidth}{0.3in}{0.84in}
\textcolor[HTML]{0D0D0D}{The web container maintains the life cycle of a servlet instance. Let's see the life cycle of the servlet:}\par

\end{adjustwidth}

\begin{enumerate}[label*={\fontsize{12pt}{12pt}\selectfont \arabic*.}]
	\item \textcolor[HTML]{0D0D0D}{Servlet class is loaded.}\par

	\item \textcolor[HTML]{0D0D0D}{Servlet instance is created.}\par

	\item \textcolor[HTML]{0D0D0D}{init method is invoked.}\par

	\item \textcolor[HTML]{0D0D0D}{service method is invoked.}\par

	\item \textcolor[HTML]{0D0D0D}{destroy method is invoked.}
\end{enumerate}\par

\begin{adjustwidth}{0.3in}{0.84in}
\subsubsection*{ }
\addcontentsline{toc}{subsubsection}{ }
\end{adjustwidth}

\begin{adjustwidth}{0.3in}{0.84in}
\subsubsection*{MySQL}
\addcontentsline{toc}{subsubsection}{MySQL}
\end{adjustwidth}


\vspace{\baselineskip}
\begin{adjustwidth}{0.3in}{0.84in}
\begin{justify}
\textcolor[HTML]{0D0D0D}{The data in a MySQL database are stored in tables. A table is a collection of related data, and it consists of columns and rows.}
\end{justify}\par

\end{adjustwidth}


\vspace{\baselineskip}
\begin{adjustwidth}{0.3in}{0.84in}
\begin{justify}
\textbf{\textcolor[HTML]{0D0D0D}{QUERIES of MySQL:-}}
\end{justify}\par

\end{adjustwidth}


\vspace{\baselineskip}
\begin{adjustwidth}{0.3in}{0.84in}
\begin{justify}
\textcolor[HTML]{0D0D0D}{CREATE <database name>;}
\end{justify}\par

\end{adjustwidth}

\begin{adjustwidth}{0.3in}{0.84in}
\begin{justify}
\textcolor[HTML]{0D0D0D}{USE <database name>;}
\end{justify}\par

\end{adjustwidth}

\begin{adjustwidth}{0.3in}{0.84in}
\begin{justify}
\textcolor[HTML]{0D0D0D}{CREATE <table name>(<col\_name> type );}
\end{justify}\par

\end{adjustwidth}

\begin{adjustwidth}{0.3in}{0.84in}
\begin{justify}
\textcolor[HTML]{0D0D0D}{SELECT  <col> FROM <table name>;}
\end{justify}\par

\end{adjustwidth}


\vspace{\baselineskip}
\begin{adjustwidth}{0.3in}{0.84in}
\begin{justify}
\textcolor[HTML]{0D0D0D}{INSERT INTO table\_name (column1, column2, column3,...)}
\end{justify}\par

\end{adjustwidth}

\begin{adjustwidth}{0.3in}{0.84in}
\begin{justify}
\textcolor[HTML]{0D0D0D}{VALUES (value1, value2, value3,...);}
\end{justify}\par

\end{adjustwidth}



 %%%%%%%%%%%%  Starting New Page here %%%%%%%%%%%%%%

\newpage

\vspace{\baselineskip}
\vspace{\baselineskip}

\vspace{\baselineskip}

\vspace{\baselineskip}
\begin{adjustwidth}{0.3in}{0.84in}
\begin{FlushRight}
{\fontsize{24pt}{28.8pt}\selectfont \textbf{\textcolor[HTML]{0D0D0D}{Testing}}\par}
\end{FlushRight}\par

\end{adjustwidth}


\vspace{\baselineskip}

\vspace{\baselineskip}
\vspace{\baselineskip}

\vspace{\baselineskip}
\begin{adjustwidth}{0.3in}{0.84in}
{\fontsize{14pt}{16.8pt}\selectfont \textbf{\textcolor[HTML]{0D0D0D}{7.1\par} }}{\fontsize{14pt}{16.8pt}\selectfont \textbf{\textcolor[HTML]{0D0D0D}{Basic Terminology}}\par}\par

\end{adjustwidth}

\begin{adjustwidth}{0.3in}{0.84in}
\textcolor[HTML]{0D0D0D}{In this research work a study has been done on 1st year student in their programming labs. In that it has been seen that most of the time students are just stuck on the compile time errors and are not able to develop their logical ability towards programming. Errors which are made by students are as following-}\par

\end{adjustwidth}

\begin{enumerate}
	\item \textcolor[HTML]{0D0D0D}{Incorrect usage of tags}\par

	\item \textcolor[HTML]{0D0D0D}{Incorrect function}\par

	\item \textcolor[HTML]{0D0D0D}{Connectivity in MySql.}\par

	\item \textcolor[HTML]{0D0D0D}{Path not defined}
\end{enumerate}\par


\vspace{\baselineskip}
\begin{adjustwidth}{0.3in}{0.84in}
\textbf{\textcolor[HTML]{0D0D0D}{Time Analysis- Usually when these records are made manually the time consumption as well as work load is high. But with web application, they will be able to make the records easily. Time taken is also very low.}}\par

\end{adjustwidth}


\vspace{\baselineskip}
\subsection{Purpose of the document}

\vspace{\baselineskip}
\begin{adjustwidth}{0.3in}{0.84in}
\textcolor[HTML]{0D0D0D}{The purpose of this document is to reduce the work load, time consumption, manual work and to increase the efficiency, store information in proper format.}\par

\end{adjustwidth}


\vspace{\baselineskip}
\subsection{Application Overview}

\vspace{\baselineskip}
\begin{adjustwidth}{0.3in}{0.84in}
\textcolor[HTML]{0D0D0D}{To track the stationary stock and bills that has been generated previously. To keep data in proper organized way and provide greater efficiency to the users of this application.}\par

\end{adjustwidth}


\vspace{\baselineskip}
\subsection{Testing of}

\vspace{\baselineskip}
\begin{itemize}
	\item \textcolor[HTML]{0D0D0D}{Internal CSS}\par

	\item \textcolor[HTML]{0D0D0D}{External CSS}\par

	\item \textcolor[HTML]{0D0D0D}{JavaScript Function}
\end{itemize}\par


\vspace{\baselineskip}

\vspace{\baselineskip}

\vspace{\baselineskip}
\begin{itemize}
	\item {\fontsize{14pt}{16.8pt}\selectfont \textbf{\textcolor[HTML]{0D0D0D}{Testing Scope}}\par}\par


\vspace{\baselineskip}
\begin{itemize}
	\item {\fontsize{13pt}{15.6pt}\selectfont \textbf{\textcolor[HTML]{0D0D0D}{In Scope}}\par}\par


\vspace{\baselineskip}\begin{itemize}
	\item {\fontsize{13pt}{15.6pt}\selectfont \textcolor[HTML]{0D0D0D}{I/O operaton}\par}\par

	\item {\fontsize{13pt}{15.6pt}\selectfont \textcolor[HTML]{0D0D0D}{Header file correction}\par}\par

	\item {\fontsize{13pt}{15.6pt}\selectfont \textcolor[HTML]{0D0D0D}{Main function correction}\par}\par

	\item {\fontsize{13pt}{15.6pt}\selectfont \textcolor[HTML]{0D0D0D}{Braces of main() function}\par}\par

	\item {\fontsize{13pt}{15.6pt}\selectfont \textcolor[HTML]{0D0D0D}{Braces of if condition}\par}\par

	\item {\fontsize{13pt}{15.6pt}\selectfont \textcolor[HTML]{0D0D0D}{Syntax of if else}\par}\par

	\item {\fontsize{13pt}{15.6pt}\selectfont \textcolor[HTML]{0D0D0D}{Syntax within for loop}\par}\par

	\item {\fontsize{13pt}{15.6pt}\selectfont \textcolor[HTML]{0D0D0D}{All the curly braces}\par}\par

	\item {\fontsize{13pt}{15.6pt}\selectfont \textcolor[HTML]{0D0D0D}{Undeclared Variable}\par}
\end{itemize}\par


\vspace{\baselineskip}

\vspace{\baselineskip}
	\item {\fontsize{13pt}{15.6pt}\selectfont \textbf{\textcolor[HTML]{0D0D0D}{Out of Scope}}\par}\par

\textcolor[HTML]{0D0D0D}{Performance Testing was not done for this application.}\par


\vspace{\baselineskip}

\vspace{\baselineskip}
	\item {\fontsize{13pt}{15.6pt}\selectfont \textbf{\textcolor[HTML]{0D0D0D}{Items not Tested}}\par}
\end{itemize}
\end{itemize}\par


\vspace{\baselineskip}\begin{itemize}
	\item {\fontsize{13pt}{15.6pt}\selectfont \textcolor[HTML]{0D0D0D}{Function calling}\par}\par

	\item {\fontsize{13pt}{15.6pt}\selectfont \textcolor[HTML]{0D0D0D}{Undeclared variables in i/o function.}\par}
\end{itemize}\par


\vspace{\baselineskip}

\vspace{\baselineskip}
\begin{adjustwidth}{-0.1in}{0.84in}
\begin{FlushRight}
{\fontsize{24pt}{28.8pt}\selectfont \textbf{\textcolor[HTML]{0D0D0D}{Process of Deployment of the}}\par}
\end{FlushRight}\par

\end{adjustwidth}

\begin{adjustwidth}{-0.1in}{0.84in}
\begin{FlushRight}
{\fontsize{24pt}{28.8pt}\selectfont \textbf{\textcolor[HTML]{0D0D0D}{Project}}\par}
\end{FlushRight}\par

\end{adjustwidth}


\vspace{\baselineskip}

\vspace{\baselineskip}
\vspace{\baselineskip}

\vspace{\baselineskip}
\subsection{Domain Purchasing Process}
\begin{adjustwidth}{1.18in}{0.55in}
 \tabto{0.59in} \textcolor[HTML]{0D0D0D}{Every website needs a catchy address. And to get one, you need to know how to buy a domain name. Luckily, domain registration has become quite a simple procedure nowadays. It’s also\ one of the first steps you need to take when you  start a blog or create a website.}\par

\end{adjustwidth}


\vspace{\baselineskip}\begin{adjustwidth}{1.18in}{0.0in}
 \tabto{0.59in} \textcolor[HTML]{0D0D0D}{The short steps to buy a domain name are:}\par

\end{adjustwidth}

\begin{itemize}
	\item {\fontsize{13pt}{15.6pt}\selectfont \textcolor[HTML]{0D0D0D}{Choose a reliable domain registrar (like Hostinger or GoDaddy).}\par}\par

	\item {\fontsize{13pt}{15.6pt}\selectfont \textcolor[HTML]{0D0D0D}{Find a domain availability checker tool.}\par}\par

	\item {\fontsize{13pt}{15.6pt}\selectfont \textcolor[HTML]{0D0D0D}{Run a domain name search.}\par}\par

	\item {\fontsize{13pt}{15.6pt}\selectfont \textcolor[HTML]{0D0D0D}{Pick the best available option.}\par}\par

	\item {\fontsize{13pt}{15.6pt}\selectfont \textcolor[HTML]{0D0D0D}{Finalise your order and complete the domain registration.}\par}\par

	\item {\fontsize{13pt}{15.6pt}\selectfont \textcolor[HTML]{0D0D0D}{Verify the ownership of your new domain.}\par}\par

 \tabto{0.59in} \textcolor[HTML]{0D0D0D}{Below, we’ll overview each step in a more in-depth fashion and present a few tips and tricks to ease the whole process.}\par


\vspace{\baselineskip} \tabto{0.59in} \textcolor[HTML]{0D0D0D}{How to Buy a Domain Name (From Hostinger)}\par

 \tabto{0.59in} \textcolor[HTML]{0D0D0D}{Now that you know how to pick a proper domain name, you might be wondering how to purchase one.}\par

 \tabto{0.59in} \textcolor[HTML]{0D0D0D}{To get a domain name for your website, you’ll need an ICANN accredited registrar (such as us). Depending on your chosen domain extension, the registration fee can range between $\$$ 0.99 to $\$$ 92.99.}\par


\vspace{\baselineskip} \tabto{0.59in} \textbf{\textcolor[HTML]{0D0D0D}{Below are 5 steps that cover the domain registration process in more depth.}}\par


\vspace{\baselineskip} \tabto{0.59in} \textcolor[HTML]{0D0D0D}{Step 1 – Find a Domain Checker}\par



%%%%%%%%%%%%%%%%%%%% Figure/Image No: 7 starts here %%%%%%%%%%%%%%%%%%%%

\begin{figure}[H]
\advance\leftskip 2.27in		\includegraphics[width=3.84in,height=1.07in]{./media/image7.jpeg}
\end{figure}


%%%%%%%%%%%%%%%%%%%% Figure/Image No: 7 Ends here %%%%%%%%%%%%%%%%%%%%

\textcolor[HTML]{0D0D0D}{The journey of buying a domain name starts with an availability lookup. In fact, we have the perfect tool to check domain vacancy here on Hostinger.}\par


\vspace{\baselineskip}
\vspace{\baselineskip} \tabto{0.59in} \textcolor[HTML]{0D0D0D}{Step 2 – Run a Domain Name Search}\par

 \tabto{0.59in} \textcolor[HTML]{0D0D0D}{Now enter your desired name in the search field and take it for a spin. The domain checker tool will present you with a list of available options that you can register.}\par


\vspace{\baselineskip}

%%%%%%%%%%%%%%%%%%%% Figure/Image No: 8 starts here %%%%%%%%%%%%%%%%%%%%

\begin{figure}[H]
	\begin{Center}
		\includegraphics[width=3.25in,height=1.95in]{./media/image8.jpeg}
	\end{Center}
\end{figure}


%%%%%%%%%%%%%%%%%%%% Figure/Image No: 8 Ends here %%%%%%%%%%%%%%%%%%%%

\par


\vspace{\baselineskip}
\vspace{\baselineskip}
\vspace{\baselineskip}
\vspace{\baselineskip}
\vspace{\baselineskip} \tabto{0.59in} \textcolor[HTML]{0D0D0D}{Step 3 – Pick Your Domain}\par

\setlength{\parskip}{0.12pt}
 \tabto{0.59in} \textcolor[HTML]{0D0D0D}{Once you locate a name that you like, proceed with the registration by pressing Add to Cart.}\par


\vspace{\baselineskip}

%%%%%%%%%%%%%%%%%%%% Figure/Image No: 9 starts here %%%%%%%%%%%%%%%%%%%%

\begin{figure}[H]
	\begin{Center}
		\includegraphics[width=3.33in,height=0.66in]{./media/image9.jpeg}
	\end{Center}
\end{figure}


%%%%%%%%%%%%%%%%%%%% Figure/Image No: 9 Ends here %%%%%%%%%%%%%%%%%%%%

\par


\vspace{\baselineskip} \tabto{0.59in} \textcolor[HTML]{0D0D0D}{If you’re not planning to look for more variations, proceed with the checkout and you’ll be able to choose a payment processor and complete your transaction.}\par


\vspace{\baselineskip} \tabto{0.59in} \textcolor[HTML]{0D0D0D}{Step 4 – Complete the Domain Registration}\par

 \tabto{0.59in} \textcolor[HTML]{0D0D0D}{As soon as you complete the domain payment, you’ll be redirected to the control panel. Inside, you’ll find the setup box to complete your domain name registration.}\par



%%%%%%%%%%%%%%%%%%%% Figure/Image No: 10 starts here %%%%%%%%%%%%%%%%%%%%

\begin{figure}[H]
\advance\leftskip 2.17in		\includegraphics[width=3.94in,height=1.69in]{./media/image10.jpeg}
\end{figure}


%%%%%%%%%%%%%%%%%%%% Figure/Image No: 10 Ends here %%%%%%%%%%%%%%%%%%%%

\par


\vspace{\baselineskip}

%%%%%%%%%%%%%%%%%%%% Figure/Image No: 11 starts here %%%%%%%%%%%%%%%%%%%%

\begin{figure}[H]
\advance\leftskip 2.29in		\includegraphics[width=3.69in,height=1.7in]{./media/image11.jpeg}
\end{figure}


%%%%%%%%%%%%%%%%%%%% Figure/Image No: 11 Ends here %%%%%%%%%%%%%%%%%%%%

\textcolor[HTML]{0D0D0D}{Make sure to fill in all the fields with the correct details, as they will be stored in the official domain ownership database called WHOIS.}\par


\vspace{\baselineskip} \tabto{0.59in} \textcolor[HTML]{0D0D0D}{After you submit your details, the domain registration will be processed and you’ll only need to follow one last step.}\par


\vspace{\baselineskip} \tabto{0.59in} \textcolor[HTML]{0D0D0D}{Step 5 – Verify the Ownership of Your New Domain}\par


\vspace{\baselineskip} \tabto{0.59in} \textcolor[HTML]{0D0D0D}{The final step of your journey is to verify the domain ownership through the email address you used while registering. It usually arrives within a few minutes after finishing the domain setup.}\par

 \tabto{0.59in} \textcolor[HTML]{0D0D0D}{In case it’s not arriving, you can re-send the request from your control panel. We recommend doing it immediately, as waiting for 15 days or more will lead to a temporary suspension from the registry.}\par



%%%%%%%%%%%%%%%%%%%% Figure/Image No: 12 starts here %%%%%%%%%%%%%%%%%%%%

\begin{figure}[H]
	\begin{Center}
		\includegraphics[width=6.66in,height=1.52in]{./media/image12.jpeg}
	\end{Center}
\end{figure}


%%%%%%%%%%%%%%%%%%%% Figure/Image No: 12 Ends here %%%%%%%%%%%%%%%%%%%%

\par


\vspace{\baselineskip}
\vspace{\baselineskip}
\end{itemize}\subsection{Server Purchasing Process.}

\vspace{\baselineskip} \tabto{0.59in} \textcolor[HTML]{0D0D0D}{Server resources tie closely to three specific types of hardware: hard disk storage; CPU size—number of cores, and to a lesser extent, clock speed; and the capacity of on-board server memory (RAM). A file server will have multiple bays for hard drives since it’s primarily used for storage. A database server that handles lots of user queries benefits from a large (12- or 16-core) CPU. Web servers and application servers have framework-specific requirements you might reference, usually the number of users querying or writing to the database affects how robust you should go with the hardware.}\par


\vspace{\baselineskip}\textcolor[HTML]{0D0D0D}{\uline{How to pick the right server for the job}}\par

{\fontsize{13pt}{15.6pt}\selectfont \textcolor[HTML]{0D0D0D}{So, getting back to that initial question—\textit{what will your server actually do}? A business buys a server to handle one or more specific tasks:}\par}\par

\begin{itemize}
	\item {\fontsize{13pt}{15.6pt}\selectfont \textcolor[HTML]{0D0D0D}{Sharing assets with file server, or network attached storage (NAS) appliance across a local network or as so-called private cloud storage. Look for: multiple hot-swappable drive bays, configurable hardware/software RAID options; a low- power CPU should suffice.}\par}\par

	\item {\fontsize{13pt}{15.6pt}\selectfont \textcolor[HTML]{0D0D0D}{Providing authentication for a domain. Username, password, levels of access, and security settings resides in a designated server computer or network switch. Called a domain controller (DC) in Windows Server, and used for managing Active Directory (AD). Look for: a virtualization-capable server (any 64-bit CPU, 4 GB+ RAM)}\par}\par

	\item {\fontsize{13pt}{15.6pt}\selectfont \textcolor[HTML]{0D0D0D}{Providing database services to other servers. Applications and websites are built upon a database layer which is often stored on its own server. Development and non-user specific tasks like data analysis, mining, archiving, and storage using Oracle, MySQL, MS Access, and similar applications utilizes this server hardware. Look for: hard drives rated for fast writes; deploy an identical backup ‘slave’ server as a read-only database.}\par}\par


\vspace{\baselineskip}	\item {\fontsize{13pt}{15.6pt}\selectfont \textcolor[HTML]{0D0D0D}{Hosting a website with a web server. Web servers use HTTP to serve files that make up web pages served to users browsing a website. Web servers work in tandem with a database server. This may occur within the same physical hardware server, or by using two servers networked together. Look for: hardware redundancy especially if you host e-commerce. Increasing server RAM capacity benefits performance under load.}\par}\par

	\item {\fontsize{13pt}{15.6pt}\selectfont \textcolor[HTML]{0D0D0D}{Providing e-mail services with a mail server. Messaging servers, like Microsoft Exchange, use specific protocols (SMTP, POP3, IMAP) to send and receive messages. Dedicating server hardware to this task is recommend for optimal operation. Look for: similar specifications as a file server.}\par}\par

	\item {\fontsize{13pt}{15.6pt}\selectfont \textcolor[HTML]{0D0D0D}{Controlling shared peripheral equipment, like a printer. Low-power specs will suffice. You might repurpose and old PC as a print server if you have one.}\par}\par

	\item {\fontsize{13pt}{15.6pt}\selectfont \textcolor[HTML]{0D0D0D}{Running shared softwareon an application server. Centralizing applications their native framework (Java, PHP, .NET, various flavors of .js) improves performance under heavy usage, makes updates easier, and reduces TCO for maintaining tools organizations use for productivity. Look for: enterprise-grade storage bays (SAS hard drives) and ECC RAM. Note that un-virtualized instances tend to work better for development.}\par}\par


\vspace{\baselineskip}\textcolor[HTML]{0D0D0D}{\uline{Choosing server form factor to fit your physical space}}\par

\textcolor[HTML]{0D0D0D}{Servers come in several different physical form factors that can be classified into three umbrellas: tower, blade, and rackmount. The form factors are determined by the server case; you’ll find the same components on the inside of comparable models.}\par

	\item {\fontsize{13pt}{15.6pt}\selectfont \textcolor[HTML]{0D0D0D}{Tower – A tower server resembles a regular desktop computers—except that they have server components inside. Same as their PC cousins, towers come in several different shapes. These make sense as first servers because they can offer plenty of processing power and don’t require you to purchase additional mounting hardware. The drawback of tower servers is that they take up more room than either rackmount or blade setups once you start adding more.}\par}\par

	\item {\fontsize{13pt}{15.6pt}\selectfont \textcolor[HTML]{0D0D0D}{Rackmount –Rackmount servers need to be installed onto a rack chassis. A chassis, typically several feet high, can hold multiple servers on top of each other in slots. Consider rackmount units when you have several servers and want to consolidate them into a smaller space.}\par}\par

	\item {\fontsize{13pt}{15.6pt}\selectfont \textcolor[HTML]{0D0D0D}{Blade\ – Similar to rackmount servers in that they require a chassis to be  installed. Blade servers are even more space-efficient than rackmount servers. However, properly\ cooling blade servers can be more challenging; consider  these when your server closet scales into a server room. They are an even bigger investment than rackmount servers.}\par}\par


\vspace{\baselineskip}
\end{itemize}\subsection{Hosting your website on the server}
 \tabto{0.59in} \textcolor[HTML]{0D0D0D}{You can buy your domain and web-hosting from multiple providers, including HostGator India. But did you know that HostGator India’s award-winning hosting services are ideal for pros and beginners with little or no technical skills? They offer you a super easy way to set up (with just one click) and manage your website}\par

 \tabto{0.59in} \textcolor[HTML]{0D0D0D}{cost-effectively with:}\par

\begin{itemize}
	\item {\fontsize{13pt}{15.6pt}\selectfont \textcolor[HTML]{0D0D0D}{Round the clock support}\par}\par

	\item {\fontsize{13pt}{15.6pt}\selectfont \textcolor[HTML]{0D0D0D}{Unlimited2 storage}\par}\par

	\item {\fontsize{13pt}{15.6pt}\selectfont \textcolor[HTML]{0D0D0D}{Unlimited (professional) email addresses}\par}\par

	\item {\fontsize{13pt}{15.6pt}\selectfont \textcolor[HTML]{0D0D0D}{(with Unlimited Autoresponders, Mail Forwards, Email Aliases, Mailing Lists, etc.)}\par}\par

	\item {\fontsize{13pt}{15.6pt}\selectfont \textcolor[HTML]{0D0D0D}{1-click installation}\par}\par

	\item {\fontsize{13pt}{15.6pt}\selectfont \textcolor[HTML]{0D0D0D}{(WordPress, Joomla, Magento, Drupal, phpBB, Gallery and many other CMSs)}\par}\par

	\item {\fontsize{13pt}{15.6pt}\selectfont \textcolor[HTML]{0D0D0D}{Latest cPanel }\par}\par

{\fontsize{14pt}{16.8pt}\selectfont \textbf{\textcolor[HTML]{0D0D0D}{Steps to Host a Wesite}}\par}\par

\textcolor[HTML]{0D0D0D}{\uline{Step 1: Decide What Type of Website You Want }}\par

\textcolor[HTML]{0D0D0D}{You will typically find 2 types of websites:}\par

	\item {\fontsize{13pt}{15.6pt}\selectfont \textcolor[HTML]{0D0D0D}{Static or Basic Websites: Static websites are simple websites with one or more web pages (called HTML pages). You can build them on your computer with software like Dreamweaver and then upload the pages to your host’s server using any FTP software (such as FileZilla).Whenever you need to make changes to your website, you’ll have to edit the pages on your computer and upload them again. Since they cannot be modified dynamically, such websites are called static websites.Static websites are cheaper than dynamic websites (below) but come with limited functionality and no option for e- commerce or interactivity.}\par}\par

	\item {\fontsize{13pt}{15.6pt}\selectfont \textcolor[HTML]{0D0D0D}{Dynamic Websites: Dynamic websites contain information that changes, depending on the time of day, the viewer and other factors. They make use of both client-side and server-side scripts to create and update content. Client- side scripts, which run on a user’s computer, are mainly used for appearance and interaction purposes. Server-side scripts, which reside on a server and are extensively used by E-commerce and social networking sites, allow users to have individual accounts and provide a customised response for each user. Dynamic websites are CMS-driven, and allow you to directly add and edit content (i.e. text, design, photos, and videos), as well as let your visitors leave comments and start discussions.Dynamic websites are ideal for businesses and organisations. Examples of dynamic websites include blogs, forums, photo galleries and e-commerce sites.}\par}\par


\vspace{\baselineskip} \tabto{0.59in} \textcolor[HTML]{0D0D0D}{Installing a web application software like WordPress, Joomla, Magento, etc. may sound complicated but it’s not. HostGator India allows you one-click installation of web applications and provides friendly 24/7/365 support to make it easy.}\par


\vspace{\baselineskip}\textcolor[HTML]{0D0D0D}{\uline{Step 2: Choose Your Hosting Server}}\par

\textcolor[HTML]{0D0D0D}{Unlike\ static\ HTML sites which can be hosted on most web servers, when it   comes to web applications, there are basically two types of hosting platforms. Depending on your hosting needs and what you’re most comfortable with, you can choose from:}\par

	\item {\fontsize{13pt}{15.6pt}\selectfont \textcolor[HTML]{0D0D0D}{Linux Hosting, which allows running scripts written in PHP, Perl, Python and other Unix-originated languages, and usually supports PostgreSQL and MySQL databases. This is the most commonly used system today.}\par}\par

	\item {\fontsize{13pt}{15.6pt}\selectfont \textcolor[HTML]{0D0D0D}{Windows Hosting, which allows running ASP scripts utilizing .NET and other Microsoft technologies, and supports Microsoft SQL Server and Access database.}\par}\par

\textcolor[HTML]{0D0D0D}{You can go with either Linux hosting or Windows hosting, regardless of which operating system you use at home or at work. If your website doesn’t require any scripting support, you’ll find Linux hosting more cost-effective. But if your website needs scripting and database support, choose the platform that supports the technologies you use.}\par


\vspace{\baselineskip}\textcolor[HTML]{0D0D0D}{\uline{Step 3: Select Your Web Hosting Plan}}\par

\textcolor[HTML]{0D0D0D}{You will typically find a wide range of services in web hosting, such as:}\par

	\item {\fontsize{13pt}{15.6pt}\selectfont \textcolor[HTML]{0D0D0D}{Shared Hosting: In shared hosting, you get to share the physical server with other website owners. However, you will have your own separate account (secured with login credentials). Shared hosting is very affordable because the cost of operating the server is shared between you and the other website owners.}\par}\par

	\item {\fontsize{13pt}{15.6pt}\selectfont \textcolor[HTML]{0D0D0D}{VPS Hosting (Virtual Private Server Hosting): In VPS hosting, every website is stored on a very powerful server that is divided into several virtual compartments. The server software is configured separately so that each unit can function independently. It should be your preferred option if you have high-security concerns but don’t want to invest in a faster (but costlier) dedicated server.}\par}\par

	\item {\fontsize{13pt}{15.6pt}\selectfont \textcolor[HTML]{0D0D0D}{Dedicated Hosting: Dedicated hosting offers you an entire server for yourself, thereby making it faster, more secure$ \ldots $ and costlier. It is the ideal solution for larger businesses and high-traffic websites because it allows for maximum customisation, configuration, installation and flexibility.}\par}\par

	\item {\fontsize{13pt}{15.6pt}\selectfont \textcolor[HTML]{0D0D0D}{Cloud Hosting: Cloud hosting allows multiple virtual servers (clouds) to work together to host a website or a group of websites. It offers unlimited ability to handle sudden traffic spikes. A cloud-hosted website is not limited to a single server, and the resources allocated to it can shrink or expand dynamically, depending on how much traffic you get. It’s a great option for large websites, including e-commerce websites, newsletters and blogs.}\par}
\end{itemize}\par

\begin{adjustwidth}{1.18in}{0.57in}
 \tabto{0.59in} \textcolor[HTML]{0D0D0D}{Most people start with VPS (or even shared) hosting and upgrade later as their business grows. VPS hosting gives you professional web hosting capabilities at a far lower price than a dedicated server.}\par

\end{adjustwidth}


\vspace{\baselineskip}\begin{adjustwidth}{1.18in}{0.0in}
\textcolor[HTML]{0D0D0D}{\uline{Step 4: Change Your DNS Address}}\par

\end{adjustwidth}

\begin{adjustwidth}{1.18in}{0.7in}
 \tabto{0.59in} \textcolor[HTML]{0D0D0D}{After you have purchased your web hosting, you will get Name Servers (also known as Domain Name Servers or DNS) – which is the Internet’s equivalent of a phone book that contains IP Addresses3.}\par

\end{adjustwidth}


\vspace{\baselineskip}\begin{adjustwidth}{1.18in}{0.57in}
\textcolor[HTML]{0D0D0D}{To get your website up and working, you will need to change the Name Servers of your domain. It’s a simple but mandatory step for you to get started.}\par

\end{adjustwidth}

	\item \textcolor[HTML]{0D0D0D}{Go to your Domain Control Panel via \href{http://manage.hostgator.in/customer}{http://manage.hostgator.in/customer.}}\par

	\item \textcolor[HTML]{0D0D0D}{Enter your registered email address and password.}\par

	\item \textcolor[HTML]{0D0D0D}{Click on the Domain Name for which you need to change the Name Servers.}\par

	\item \textcolor[HTML]{0D0D0D}{In the Domain Registration section, click on the Name Servers option.}\par

	\item \textcolor[HTML]{0D0D0D}{Replace the existing Name Servers with the ones provided by your current web host, and click on the Update Name Servers button.}\par

 \tabto{0.59in} \textcolor[HTML]{0D0D0D}{If you have registered your domain name with a third party provider, you\ will need  to log in to their Control Panel, update the Name Servers of the domain to those provided by HostGator. However, if your domain is already using the Name Servers of the third party provider, you can add an A Record for the domain pointing to HostGator’s Server IP in the third Party DNS Zone.\tab }\par

 \tabto{0.59in} \textcolor[HTML]{0D0D0D}{After you have changed your DNS, it will take about 24-48 hours for your website to start resolving to HostGator India’s servers.}\par


\vspace{\baselineskip}\textcolor[HTML]{0D0D0D}{\uline{Step 5: Upload Your Website}}\par

\textcolor[HTML]{0D0D0D}{You can now upload your website to your account by connecting to the server using either cPanel’s File Manager or FTP Client (such as FileZilla) – after which your website will go live.}\par

\textcolor[HTML]{0D0D0D}{How to Upload Your Website Using cPanel File Manager}\par

\begin{itemize}
	\item \textcolor[HTML]{0D0D0D}{Log in to your cPanel.}\par

	\item \textcolor[HTML]{0D0D0D}{Click on the icon titled File Manager.}\par

	\item \textcolor[HTML]{0D0D0D}{Select Web Root and click on Go.}\par

	\item \textcolor[HTML]{0D0D0D}{Add all the files and folders under public\_html and their respective domain folder.}
\end{itemize}\par


\vspace{\baselineskip}\begin{adjustwidth}{1.18in}{0.0in}
\textcolor[HTML]{0D0D0D}{\uline{How to Upload Your Website Using FTP Client}}\par

\end{adjustwidth}

\begin{adjustwidth}{0.98in}{0.57in}
\textcolor[HTML]{0D0D0D}{You can connect to FTP via an FTP program such as FileZilla Client. It allows you to see the files and folders on our server like you’d see them on your computer.}\par

\end{adjustwidth}

\begin{adjustwidth}{1.18in}{0.7in}
\textcolor[HTML]{0D0D0D}{You can use it to drag and drop your website’s files into the /public\_html/ folder. To connect to your web server via FileZilla, follow these steps:}\par

\end{adjustwidth}

\begin{itemize}
	\item \textcolor[HTML]{0D0D0D}{Install FileZilla and open it}\par

	\item \textcolor[HTML]{0D0D0D}{From the File menu, select Site Manager}\par

	\item \textcolor[HTML]{0D0D0D}{Click on New Site}\par

	\item \textcolor[HTML]{0D0D0D}{Name the New Site – such as with your real domain name}\par

	\item \textcolor[HTML]{0D0D0D}{Enter your website’s IP address in the field marked FTP Address}\par

	\item \textcolor[HTML]{0D0D0D}{Enter the username and password you received in your welcome mail}\par

	\item \textcolor[HTML]{0D0D0D}{Set the Port to 21 (FTP always runs on Port 21)}\par

	\item \textcolor[HTML]{0D0D0D}{Click Connect}
\end{itemize}\par


\vspace{\baselineskip}\begin{adjustwidth}{1.18in}{0.0in}
\textcolor[HTML]{0D0D0D}{Once your FTP is connected, you will see the files and folders of your:}\par

\end{adjustwidth}

	\item \textcolor[HTML]{0D0D0D}{Local computer on the left}\par

	\item \textcolor[HTML]{0D0D0D}{Web hosting service on the right}\par

\textcolor[HTML]{0D0D0D}{To upload files to your hosting service provider via FileZilla, follow these steps:}\par

\begin{itemize}
	\item \textcolor[HTML]{0D0D0D}{From the left-hand side of FileZilla, select the file(s) and folder(s) you want to upload.}\par


\vspace{\baselineskip}	\item \textcolor[HTML]{0D0D0D}{Drag and drop the file(s) and folder(s) to the directory location on the right side of your web hosting service. FileZilla will now start uploading.}\par

	\item \textcolor[HTML]{0D0D0D}{After the uploading is finished, FileZilla log will confirm success and your uploads will be visible on the right-hand side}
\end{itemize}\par


\vspace{\baselineskip}\subsection{Accessing process of online project through URL}

\vspace{\baselineskip}\begin{adjustwidth}{1.18in}{0.62in}
\textcolor[HTML]{0D0D0D}{The Project Online Desktop client is included with your Project Online Professional or Project Online Premium license. While you can use it as a standalone client to create and manage your project plans offline, you can also use it to connect to Project Online in your Office 365 environment to work with your Project Online users. For example, you can create, save, and publish your projects to Project Online, and team members assigned to your project tasks can use Project Online to give you updates on their task status.}\par

\end{adjustwidth}


\vspace{\baselineskip}\begin{adjustwidth}{1.18in}{0.0in}
\textcolor[HTML]{0D0D0D}{\uline{How to connect to Project Online}}\par

\end{adjustwidth}

	\item \textcolor[HTML]{0D0D0D}{ After opening the Project Online Desktop Client, at the login screen, for Profile select Computer, and then select OK.}\par



%%%%%%%%%%%%%%%%%%%% Figure/Image No: 13 starts here %%%%%%%%%%%%%%%%%%%%

\begin{figure}[H]
\advance\leftskip 2.29in		\includegraphics[width=3.71in,height=1.69in]{./media/image13.jpeg}
\end{figure}


%%%%%%%%%%%%%%%%%%%% Figure/Image No: 13 Ends here %%%%%%%%%%%%%%%%%%%%

\par


\vspace{\baselineskip}
\vspace{\baselineskip}	\item \textcolor[HTML]{0D0D0D}{On the next screen, select Blank project.}\par

	\item \textcolor[HTML]{0D0D0D}{On the new project page, select the File menu.}\par

	\item \textcolor[HTML]{0D0D0D}{On the Backstage menu, select Info, and then select Manage Accounts.}\par



%%%%%%%%%%%%%%%%%%%% Figure/Image No: 14 starts here %%%%%%%%%%%%%%%%%%%%

\begin{figure}[H]
\advance\leftskip 2.1in		\includegraphics[width=4.07in,height=1.85in]{./media/image14.png}
\end{figure}


%%%%%%%%%%%%%%%%%%%% Figure/Image No: 14 Ends here %%%%%%%%%%%%%%%%%%%%

\par


\vspace{\baselineskip}	\item \textcolor[HTML]{0D0D0D}{On the Project Web Apps Accounts page, select Add.}\par

	\item \textcolor[HTML]{0D0D0D}{On the Account Properties page:}\par

	\item \textcolor[HTML]{0D0D0D}{For Account Name, type a name for this profile.}\par


\vspace{\baselineskip}	\item \textcolor[HTML]{0D0D0D}{For Project Server URL type the URL for your Project Web App home page in Project Online. Check with your Office 365 admin if you do not know what it is.}\par

	\item \textcolor[HTML]{0D0D0D}{Select Set as default account if you want to use this as your default profile each time you open Project Online Desktop Client.}\par

	\item \textcolor[HTML]{0D0D0D}{Click OK.}\par



%%%%%%%%%%%%%%%%%%%% Figure/Image No: 15 starts here %%%%%%%%%%%%%%%%%%%%

\begin{figure}[H]
\advance\leftskip 2.28in		\includegraphics[width=3.71in,height=1.99in]{./media/image15.jpeg}
\end{figure}


%%%%%%%%%%%%%%%%%%%% Figure/Image No: 15 Ends here %%%%%%%%%%%%%%%%%%%%

\par


\vspace{\baselineskip}

%%%%%%%%%%%%%%%%%%%% Figure/Image No: 16 starts here %%%%%%%%%%%%%%%%%%%%

\begin{figure}[H]
	\begin{FlushLeft}		\includegraphics[width=3.28in,height=1.71in]{./media/image16.jpeg}
	\end{FlushLeft}\end{figure}


%%%%%%%%%%%%%%%%%%%% Figure/Image No: 16 Ends here %%%%%%%%%%%%%%%%%%%%

	\item \textcolor[HTML]{0D0D0D}{Your new account will now show On the Project Web App Accounts page. Click OK.}\par


\vspace{\baselineskip}
\vspace{\baselineskip}	\item \textcolor[HTML]{0D0D0D}{ Close and then reopen the Project Online Desktop Client. At the login window, select your account and click OK to connect to Project Online.}\par



%%%%%%%%%%%%%%%%%%%% Figure/Image No: 17 starts here %%%%%%%%%%%%%%%%%%%%

\begin{figure}[H]
\advance\leftskip 2.71in		\includegraphics[width=3.71in,height=1.79in]{./media/image17.png}
\end{figure}


%%%%%%%%%%%%%%%%%%%% Figure/Image No: 17 Ends here %%%%%%%%%%%%%%%%%%%%

\par


\vspace{\baselineskip}

\vspace{\baselineskip}
 \tabto{0.59in} 
\vspace{\baselineskip}
\vspace{\baselineskip}

\vspace{\baselineskip}

\vspace{\baselineskip}

\vspace{\baselineskip}

\vspace{\baselineskip}
{\fontsize{24pt}{28.8pt}\selectfont \textbf{\textcolor[HTML]{0D0D0D}{CONCLUSION}}\par}\par


\vspace{\baselineskip}

\vspace{\baselineskip}
\vspace{\baselineskip}

\vspace{\baselineskip}
\begin{justify}
\textcolor[HTML]{0D0D0D}{On understanding the requirement of a software requirement by the college stationary management system we made a web portal for it. For this web portal we firstly we understood the process done by our college and then proceeded to make a web portal for it. This web portal was made to provide a software application for the department of the college, to reduce clenical work as most of the things will accomplished automatically with the help of the application software. }
\end{justify}\par

\begin{justify}
\textcolor[HTML]{0D0D0D}{Our Stationary Management System provides application software to carry out activities regarding stationary products for the college which is an automated system instead of making manual entries into books which consumes a great amount of time.}
\end{justify}\par

\begin{justify}
\textcolor[HTML]{0D0D0D}{The project intends to introduce a user friendly application software for various related activities such as record purchase order from staff members, stationary product requisition, generate purchase orders for suppliers and generate bills for the accounts section and many other related processes.}
\end{justify}\par


\vspace{\baselineskip}
\begin{justify}
\textbf{\textcolor[HTML]{0D0D0D}{Advantages of the project:}}
\end{justify}\par

\begin{itemize}
	\item \textcolor[HTML]{0D0D0D}{Efficiency is increased. }\par

	\item \textcolor[HTML]{0D0D0D}{Work load is decreased.}\par

	\item \textcolor[HTML]{0D0D0D}{Time consumption is decreased.}\par

	\item \textcolor[HTML]{0D0D0D}{User friendly application.}\par

	\item \textcolor[HTML]{0D0D0D}{Open source application.}\par

	\item \textcolor[HTML]{0D0D0D}{Secure database.}\par

	\item \textcolor[HTML]{0D0D0D}{Simple coding techniques.}\par

	\item \textcolor[HTML]{0D0D0D}{Easy maintenance.}
\end{itemize}\par


\vspace{\baselineskip}
\setlength{\parskip}{0.84pt}

\vspace{\baselineskip}

\vspace{\baselineskip}

\vspace{\baselineskip}

\vspace{\baselineskip}

\vspace{\baselineskip}

\vspace{\baselineskip}

\vspace{\baselineskip}

\vspace{\baselineskip}
\begin{adjustwidth}{3.8in}{0.84in}
{\fontsize{24pt}{28.8pt}\selectfont \textbf{\textcolor[HTML]{0D0D0D}{\  REFERENCES}}\par}\par

\end{adjustwidth}


\vspace{\baselineskip}

\vspace{\baselineskip}
\vspace{\baselineskip}

\vspace{\baselineskip}

\vspace{\baselineskip}

\vspace{\baselineskip}
	\item {\fontsize{13pt}{15.6pt}\selectfont \uline{https://docs.microsoft.com/en-us/projectonline/connect-to-project-online-with- the-project-online-desktop-client}\par}\par

	\item \href{http://www.stat.fi/meta/kas/paivapalkkainen_en.html}{{\fontsize{13pt}{15.6pt}\selectfont \ul{https://www.stat.fi/meta/kas/paivapalkkainen\_en.html}}\par}\par

	\item \href{http://www.thefreedictionary.com/wage%2Bworker}{{\fontsize{13pt}{15.6pt}\selectfont \ul{https://www}.thefr\href{http://www.thefreedictionary.com/wage%2Bworker}{eedictionary.com/wage+worker}}\par}\par

	\item {\fontsize{13pt}{15.6pt}\selectfont \uline{https://timesofindia.indiatimes.com/business/india-business/Daily-wagers-in- worker-definition/articleshow/2005951.cms}\par}\par

	\item \href{http://www.citehr.com/503925-difference-between-daily-rated-casual-}{{\fontsize{13pt}{15.6pt}\selectfont https://www}.citehr\href{http://www.citehr.com/503925-difference-between-daily-rated-casual-}{.com/503925-di}ff\href{http://www.citehr.com/503925-difference-between-daily-rated-casual-}{erence-between-daily-rated-casual-} workers.html\par}\par

	\item \href{http://www.ilo.org/wcmsp5/groups/public/---asia/---ro-bangkok/---sro-}{{\fontsize{13pt}{15.6pt}\selectfont \ul{https://www}.ilo.or\href{http://www.ilo.org/wcmsp5/groups/public/---asia/---ro-bangkok/---sro-}{g/wcmsp5/groups/public/---asia/---ro-bangkok/---sro-} new\_delhi/documents/publication/wcms\_638305.pdf}\par}\par

	\item \href{http://www.eurofound.europa.eu/observatories/eurwork/industrial-relations-}{{\fontsize{13pt}{15.6pt}\selectfont \ul{https://www}.eur\href{http://www.eurofound.europa.eu/observatories/eurwork/industrial-relations-}{ofound.europa.eu/observatories/eurwork/industrial-relations-} dictionary/casual-worker}\par}\par

	\item \href{http://www.mightyrecruiter.com/recruiter-guide/hiring-glossary-a-to-z/casual-}{{\fontsize{13pt}{15.6pt}\selectfont \ul{https://www}.mightyr\href{http://www.mightyrecruiter.com/recruiter-guide/hiring-glossary-a-to-z/casual-}{ecruiter.com/recruiter-guide/hiring-glossary-a-to-z/casual-} employment/}\par}\par

	\item \href{http://www.ilo.org/global/topics/wages/projects/lang--en/index.htm}{{\fontsize{13pt}{15.6pt}\selectfont \ul{https://www}.ilo.or\href{http://www.ilo.org/global/topics/wages/projects/lang--en/index.htm}{g/global/topics/wages/projects/lang--en/index.htm}}\par}\par

	\item \href{http://www.jstor.org/stable/2597916?seq=1}{{\fontsize{13pt}{15.6pt}\selectfont \ul{https://www}.jstor\href{http://www.jstor.org/stable/2597916?seq=1}{.org/stable/2597916?seq=1}}\par}\par

	\item {\fontsize{13pt}{15.6pt}\selectfont \uline{https://tradingeconomics.com/india/wages}\par}\par

	\item {\fontsize{13pt}{15.6pt}\selectfont \uline{https://brainly.in/question/7464259}\par}\par

	\item {\fontsize{13pt}{15.6pt}\selectfont \uline{https://economictimes.indiatimes.com/small-biz/legal/the-code-on- wages-2019-understanding-the-key-changes-to-wages-remuneration-and- bonus/articleshow/72913106.cms?from=mdr}\par}\par

	\item \href{http://www.fairwork.gov.au/employee-entitlements/types-of-employees/casual-}{{\fontsize{13pt}{15.6pt}\selectfont \ul{https://www.fairwork.gov.au/employee-entitlements/types-of-employees/casual-} part-time-and-full-time/casual-employees}\par}\par

	\item \href{http://www.dhyeyaias.com/sites/default/files/Download-Dhyeya-IAS-Perfect-7-}{{\fontsize{13pt}{15.6pt}\selectfont \ul{https://www.dhyeyaias.com/sites/default/files/Download-Dhyeya-IAS-Perfect-7-} W e e k l y - M a g a z i n e - i n - E n g l i s h - S e p t e m b e r - 2 0 1 8 - Issue-1\_www.dhyeyaias.com\_.pdf}\par}\par

	\item {\fontsize{13pt}{15.6pt}\selectfont \uline{https://reqtest.com/requirements-blog/requirements-analysis/}\par}\par

	\item \href{http://www.javatpoint.com/software-engineering-requirement-analysis}{{\fontsize{13pt}{15.6pt}\selectfont \ul{https://www.javatpoint.com/software-engineering-r}equir\href{http://www.javatpoint.com/software-engineering-requirement-analysis}{ement-analysis}}\par}\par

	\item \href{http://www.guru99.com/functional-vs-non-functional-requirements.html}{{\fontsize{13pt}{15.6pt}\selectfont \ul{https://www.guru99.com/functional-vs-non-functional-r}equir\href{http://www.guru99.com/functional-vs-non-functional-requirements.html}{ements.html}}\par}\par

	\item \href{http://www.geeksforgeeks.org/software-engineering-classification-of-software-}{{\fontsize{13pt}{15.6pt}\selectfont \ul{https://www.geeksfor}geeks.or\href{http://www.geeksforgeeks.org/software-engineering-classification-of-software-}{g/software-engineering-classification-of-software-} requirements/}\par}\par

	\item \href{http://www.techopedia.com/definition/4371/system-requirements}{{\fontsize{13pt}{15.6pt}\selectfont \ul{https://www.techopedia.com/definition/4371/system-r}equir\href{http://www.techopedia.com/definition/4371/system-requirements}{ements}}\par}\par

	\item {\fontsize{13pt}{15.6pt}\selectfont \uline{https://techterms.com/definition/system\_requirements}\par}\par

	\item \href{http://www.geeksforgeeks.org/types-software-testing/}{{\fontsize{13pt}{15.6pt}\selectfont \ul{https://www.geeksfor}geeks.or\href{http://www.geeksforgeeks.org/types-software-testing/}{g/types-software-testing/}}\par}\par


\vspace{\baselineskip}

\printbibliography
\end{document}